\chapter{Defeasible Reasoning in Formal Concept Analysis}
\label{chapter:defeasible-reasoning-in-fca}

The conclusion of \Cref{chapter:defeasible-reasoning-in-fca} marks the completion of the scaffolding of this work. The remaining chapters are dedicated to the unification of these ideas. That is, to developing an approach to defeasible reasoning in FCA in the style of the KLM framework.

Fortunately (perhaps by design), the context switch is quite natural. As we progress, the reasons behind the
selection of the KLM framework as an ideal candidate for defeasible reasoning in FCA should become apparent. Indeed,
understanding the proceeding chapter(s) is benefitted by frequently ``pattern-matching'' between the definitions presented
here, and their counterparts from prior chapters; and so we encourage---and will attempt to facilitate---these frequent backward
references.

\section{Motivation}
\label{section:motivation}

There are many perspectives one may adopt as a lense with which to understand the purpose of FCA. Certainly, some of these
perspectives should result in strong arguments against accepting non-monotonicity. Likewise, there are several, distinct,
arguments in favour. One of the most compelling reasons for the adoption of defeasibility in FCA follows from the argument
put forward in \cite{Wille2005}, where the view that concepts are frequently \textit{personal} or \textit{prototypical}
units is discussed. The recognition that defeasible reasoning facilitates a kind of prototypicality is the basis for \textcolor{red}{chapter:rational-concepts},
and so we will say no more until then.

Another argument comes to the fold when considering the utility of implications in FCA.

\begin{figure}[H]
	\centering
	\small
	\begin{cxt}
		\cxtName{\textbf{\texttt{Congressional Voting Records}}} \att{\texttt{democrat}}
		\att{\texttt{republican}} \att{\texttt{mx-missile}}
		\att{\texttt{inverse}} \att{\texttt{commutativity}}
		\obj{x....}{\texttt{magma}} \obj{xx...}{\texttt{semigroup}}
		\obj{xxx..}{\texttt{monoid}} \obj{xxxx.}{\texttt{group}}
		\obj{xxxxx}{\texttt{abelian group}} \obj{x.xx.}{\texttt{loop}}
		\obj{x..x.}{\texttt{quasigroup}} \obj{.xxx.}{\texttt{groupoid}}
		\obj{.xx..}{\texttt{category}} \obj{.x...}{\texttt{semicategory}} \end{cxt}
	\caption{A formal context showing necessary properties of group-like
		structures.} \label{context:congressional-voting-records}
\end{figure}

\subsection{Association Rules}
\label{subsection:association-rules}

% \clearpage

% \section{Preferential and Ranked Contexts}
% \clearpage

% \section{Finding order}
% \clearpage
% This phenomenon was already known to the ancient Greeks, who used the term enthymeme to refer to an argument in which
% one or more premises are left implicit. That is the idea that we develop in this section. Also called \textit{expectations}.
% \cite{makinson2003bridges}.

\begin{algo} {\textsc{ObjectRank}}

	\label{algorithm:ObjectRank}

	\Require A set $\Delta$ of defeasible conditionals over an attribute set
	$M$ \Require A formal context $\GMI$

	\Ensure A ranking $\mathsf{R} \colon G \to \mathbb{N}$ such that $\Delta$
	holds in $\RGMI$ if $\GMI$ is $\Delta$-compatible; $\bot$, otherwise.

	\State $i \coloneq 0$ \State \textit{initialise} $\mathsf{R}(g) \coloneq
		\vert G \vert$ for all $g \in G$ \State $\Gamma \coloneq \Delta$

	\While{$\exists \, g \in G \colon \mathsf{R}(g) = \vert G \vert$}
	\ForAll{$g \in G$ with $\mathsf{R}(g) = \vert G \vert$}
	\If{$g \vDash \Gamma$}
	\textit{update} $\mathsf{R}(g) = i$
	\EndIf
	\If{$\forall \, g \in G \colon \mathsf{R}(g) \neq i$}
	\State \Return $\bot$ \EndIf
	\State $\Gamma \coloneq \{(\phi \twiddle \psi) \in \Gamma \mid g \nvDash
		\phi \; \text{for all} \; g \in G \; \text{with}\; \mathsf{R}(g) = i\}$
	\State $i \coloneq i + 1$
	\EndFor
	\EndWhile

	\State \Return $\mathsf{R}$
\end{algo}

\chapter{Rational Entailment}
\begin{definition}
	A conditional $A \twiddle B$ \emph{rationally follows} from a set $\mathcal{T}$ of conditionals if and only if
	$\mathbb{T}, \mathcal{T}\dentails A \twiddle B$, where $\mathbb{T}$ is the \emph{test context}. It \emph{contextually
		follows} when $\mathbb{R}, \mathcal{T}\dentails A \rightarrow B$.
\end{definition}

\begin{definition}
	A set $\mathcal{T}$ is \emph{closed} with respect to a pair if and only if it contains every conditional that is
	rationally entailed. That is, if
	$\mathcal{T}= \{A \twiddle B \mid A,B \subseteq M \tand \mathbb{T},\mathcal{T}\dentails A \twiddle B\}$. $\mathcal{T}$
	is \emph{contextually closed} if it contains every conditional that is contextually entailed, i.e.
	$\mathcal{T}= \{A \twiddle B \mid A,B \subseteq M \tand \mathbb{R}, \mathcal{T}\dentails A \rightarrow B\}$.
\end{definition}

\begin{definition}
	A set $\mathcal{T}$ of conditionals is \emph{rationally complete} with respect to $\mathbb{T},\Delta$ if and only if
	every conditional that is in the rational closure of $\mathbb{T}, \Delta$ rationally follows from $\mathcal{T}$, i.e.,
	$\mathbb{T},\mathcal{T}$ and $\mathbb{T}, \Delta$ define the same entailment relation.
\end{definition}

Is it possible to find a $\mathcal{T}$ such that $\mathcal{T}$ is a compact form of $\Delta$. I.e., it should produce the
same ranking on $\mathbb{T}$.

\begin{definition}
	A set $\mathcal{T}$ of conditionals is \emph{contextually complete} with respect to $\mathbb{R}, \Delta$ if and only
	if every implication that is in the contextual closure of $\mathbb{R}, \Delta$ rationally follows from $\mathcal{T}$.
\end{definition}

A set of conditionals that is contextually complete to a ranked context $\mathbb{R}$ and constraint set $\Delta$
provides abstracted perspective, in terms of defeasible conditionals, on the information contained in the pair
$(\mathbb{R}, \Delta )$. It is now the goal, as was done in \Cref{subsection:implication-bases}, to find such a set that
is non-redundant, or the \textit{rational basis} of $(\mathbb{R},\Delta)$.

\section{Lexicographic Closure}


\section{C-Inference}
