\chapter{Defeasible Reasoning in Formal Concept Analysis}
\label{chapter:defeasible-reasoning-in-fca}

The conclusion of \Cref{part:1} marks the end of the scaffolding component of this work. The remaining chapters are devoted to the novel contributions of this work: the unification of the
ideas presented in \Cref{chapter:defeasible-reasoning,chapter:formal-concept-analysis}. That is, to develop an approach to defeasible reasoning in FCA in the style of rational consequence.

Fortunately, the context switch is quite natural. As we progress, the reasons behind the selection of the KLM framework as an ideal candidate for defeasible reasoning in FCA should become apparent.
Indeed, understanding the proceeding chapter(s) is benefitted by frequently ``pattern-matching'' between the definitions presented here, and their counterparts from prior chapters. So we
encourage---and will attempt to facilitate---these frequent backwards-facing references.

This chapter draws on two publications, \cite{Carr2024} and \cite{Carr2025}.

\section{Motivation}
\label{section:motivation}

The violation of monotonicitiy in FCA introduces peculiarities which can significantly undermine parts of the theory underlying FCA. For instance, one perspective on concept lattices is that
they describe chains of inheritance from concepts to their subconcepts. This idea becomes difficult to conceptualise without monotonicity.

Nevertheless, there are certain settings where violation of monotonicity extends the utility of FCA in a reasonable way. The justifications for this are somewhat similar to those given in \Cref{section:nmr-background}:
frequently there are exceptional instances of some class, and it is useful to be able to sometimes reason in a way which considers these exceptions, and at other times reason prototypically.
That is, to reason in spite of the exceptions. There is an extension of this argument \cite{Belohlavek2022,Wille2005} which makes a stronger case for the prototypical view of concepts; but
we will leave this discussion for the next chapter where it is more relevant.

Consider the context, which will be uses as a running example throughout this chapter.
%
\begin{figure}[H]
	\centering
	\begin{cxt}
		\label{cxt:voting}

		\cxtName{\textbf{\texttt{Congressional}}}

		\atr{\texttt{dem}}

		\atr{\texttt{rep}}

		\atr{\texttt{inf}}

		\atr{\texttt{wpcs}}

		\atr{\texttt{abr}}

		\atr{\texttt{pff}}

		\atr{\texttt{esa}}

		\atr{\texttt{rgs}}

		\atr{\texttt{ast}}

		\atr{\texttt{aid}}

		\atr{\texttt{mxm}}

		\atr{\texttt{imm}}

		\atr{\texttt{scc}}

		\atr{\texttt{edu}}

		\atr{\texttt{srs}}

		\atr{\texttt{crm}}

		\atr{\texttt{dfe}}

		\atr{\texttt{easa}}

		\obj{.X.X.XXX...X.XXX.X}{\texttt{moc-0}}

		\obj{.X.X.XXX.....XXX..}{\texttt{moc-1}}

		\obj{X..X.XXX.......XXX}{\texttt{moc-6}}

		\obj{X.XXX...XXX.....XX}{\texttt{moc-34}}

		\obj{ X.XXX...XXX....... }{\texttt{moc-62}}

		\obj{ .X...XX....X.XXX.. }{\texttt{moc-79}}
	\end{cxt}
	\caption{A portion of the 1984 United States Congressional Voting Records dataset}
	\label{figure:voting-records}
\end{figure}
%
For practical reasons, we show only a portion of this dataset. If the goal were the discovery of voting patterns, which characterise either party preferences or some other associations between
different bills, the brittleness of attribute implications is a limiting factor on what might be learned. The existence of a single Democrat who failed to tow the party line would prevent an
otherwise clear voting pattern from being discovered. This problem arises in other ways as well. Data might be captures incorrectly, or, in this instance, a member of congress may have been
absent, in which case there is no way to distinguish between this as opposed to a vote to reject the bill.

\subsection{Association Rules}
\label{subsection:association-rules}

Concerns over the brittleness of the semantics of (classical) attribute implications are well known in the FCA literature, and in data mining more generally \cite{Lakhal2005}. \textit{Association
rules} are able express implication-like statements despite the existence of counterexamples in the data, or in our case a context. These are statements like \say{$66\%$ of the members of congress who were \textit{Democrats}, which makes up $50\%$ of congress, voted on the \textit{MX-Missile} bill}.

The \emph{support} of a set of attributes $X \subseteq M$ in a context $\GMI$ is the fraction of objects which have all attributes $X$ over the total number of objects; and so
$\mathrm{supp}( X) = \nicefrac{|X^{\downarrow}|}{|G|}$, the (relative) support of a rule $A \Rightarrow B$ is simply the support of $A$. The \textit{confidence} of a rule is a measure of how
frequently the premise and conclusion occur, relative to how frequently the premise occurs, formally $\mathrm{conf}(A \Rightarrow B) = \nicefrac{\mathrm{supp}(X \cup Y)}{\mathrm{supp}(X)}$.
In the example above, the support of the rule is $50\%$ and its confidence is $66\%$. Association rules which meet specified lower-thresholds for support and confidence are then accepted.

While association rules present a reasonable solution to the problem of discovering and representing implications with exceptions, there are some undesirable qualities. Settling on the
threshold vallues for support and confidence is usually dependent on human input, and there is no obvious intuition for how one might arrive at these. Frequently, it is a requirement to know
the data well, which creates a kind of circularity as the purpose of this work is to extend a tool which helps one understand their data.

The situation gets worse when we try to describe the properties of reasoning with association rules. Consider the two rules below, which should be accepted with $\mathrm{minconf}= 0.6$ and
$\mathrm{minsup}= 0.5$.
%
\begin{align*}
	\texttt{dem}\Rightarrow \texttt{mxm} \\
	%; \text{with}\; \mathrm{Sup}= 0.5 \; \text{and}\; \mathrm{Conf}= 0.66 \\
	\texttt{dem}\Rightarrow \texttt{dfe} % \; \text{with}\; \mathrm{Sup}= 0.5 \; \text{and}\; \mathrm{Conf}= 0.66
\end{align*}
%
In natural language, we might summarise this scenario as saying \say{Democrats voted in favour of the \textit{MX-Missile} and \textit{Duty-Free Exports}}. To make future analysis simpler, we
could suggest that whenever we accept association rules of the form
\begin{align*}
	\phi \Rightarrow \psi    \\
	\phi \Rightarrow \gamma,
\end{align*}
we may compress the information as the simpler representation:
\begin{align*}
	\phi \Rightarrow \psi \land \gamma
\end{align*}

Unfortunately, this is not always the case. In the above scenario, the rule $\texttt{dem}\Rightarrow \{\texttt{mxm,dfe}\}$ has a confidence of $0.47$, which is below the $\mathrm{minconf}$ threshold.

That the pattern of reasoning enforced by association rules does not satisfy the above characterisation---which is the same property as was discussed at \Cref{postulate:and}---may be overlooked
in isolation. The more general problem is that association rules do not admit description via \lucas{the kind of axiomisation we want [source needed]}.

\section{Preferential Contexts}

We begin the introduction of KLM style defeasible reasoning to FCA by defining an analogue to preferential interpretations. At the start, we attempt to make as few changes to the FCA setting
as possible, resulting in some restrictions on the analogy to preferential reasoning; for example, we start with a language without negation or disjunction and so do not talk about the `or' postulate.
As we progress, we discuss the amendments to the FCA setting that necessary to lift these restrictions and get closed to the full KLM setting. The idea behind the order of this exposition is
that we view this work largely as extending FCA with ideas from preferential reasoning. As such, there may be some use cases in FCA of preferential reasoning which benefit from remaining
faithful to the original setting, and do not require the full expressivity of the KLM framework.

With that said, we define the first extension of a \textit{preferential context}.
%
\begin{definition}
	\label{definition:preferential-context} \index{formal context! preferential context}

	A \emph{preferential context} $\pcontext = \PGMI$ is a formal context $\GMI$ where $\prec$ denotes a strict partial order, or \emph{preference order}, on the set of objects.
\end{definition}
%
Preferential contexts, albeit with a different motivation, were first introduced by Obiedkov \shortcite{Obiedkov2012}, and later on with our purposes in mind in \cite{Carr2024}. They are nothing
more than a formal context with a preference relation. An example of what such a preference relation might look like, with respect to the context in \Cref{figure:voting-records}, is given
below.
%
\begin{figure}[H]
	\centering
	\begin{tikzpicture}[every node/.style={ circle, text=black, inner sep=0pt, minimum size=4mm, font=\small }]
		\node (root) at (0,0) {$\texttt{moc-0}$};
		\node (1) at (-2,1.5) {$\texttt{moc-6}$};
		\draw (root) -- (1);
		\node (2) at (0,1.5) {$\texttt{moc-1}$};
		\draw (root) -- (2);
		\node (3) at (2,1.5) {$\texttt{moc-79}$};
		\draw (root) -- (3);
		\node (12) at (-4,3) {$\texttt{moc-34}$};
		\draw (1) -- (12);
		\node (13) at (-2,3) {$\texttt{moc-62}$};
		\draw (1) -- (13);
	\end{tikzpicture}
	\caption{A preference relation on the context in \Cref{figure:voting-records}}
	\label{figure:preference-relation-congress}
\end{figure}
%
Recall that preferential interpretations were defined by a relation on states which were then mapped to valuations, with no requirement that the mapping be injective. It may then be surprising
that states seem to have been entirely omitted from our definition. The reason being that formal contexts have no restriction on the inclusion of two objects with an equivalent intension, meaning
that states seem to be implicitly constructed in contexts.

\begin{lemma}
	\label{lemma:objects-preferential}

	There exists a preferential context $\PGMI$ where there exist two objects, $g,h \in G$, with $g^{\uparrow}= h^{\uparrow}$ that defines the consequence relation $\twiddle_{\pin}$ such that
	no preferential context $\pcontext'$ without `duplicate' objects defines the same relation.
\end{lemma}

Under the preference relation, we describe the $\prec$-minimal states as one might expect, through the process of \textit{minimisation}.
\begin{definition}
	\label{definition:minimisation}

	In a preferential context $\pcontext = \PGMI$, the \emph{minimisation} of a set $A \subseteq G$ is defined by
	\[
		\underline{A}\coloneqq \{\, g \in A \mid \nexists h \in A : (h \prec g) \,\}.
	\]
\end{definition}

Frequently, the minimisation operator is applied to a set of objects that is the result of previous derivation from a set of attributes. That is, we begin with the attribute set
$\{\, \texttt{rep}\,\}$ and wish to find the most preferred objects who are democrats, $\Mind{\{\texttt{rep}\}}$. We refer to this process as the \textit{minimised-derivation}. The process
of applying another derivation operator to the result of a minimised-derivation is called the \textit{minimised-return} and denoted $\Minr{\{\texttt{rep}\}}$.

The derivation of $\{\texttt{rep}\}$ is the object set $\{\texttt{moc-0,moc-1,moc-79}\}$. The minimised-derivation, with respect to the preference conveyed by \Cref{figure:preference-relation-congress},
is the singleton $\{\texttt{moc-0}\}$. This may be understood as conveying the information that \texttt{moc-0} should be regarded as the most preferred Republican. In some sense this is correct,
however, we note some subtletly here. The status of \texttt{moc-0} as the unique minimal Republican need not be the result of some Republican-specific feature absent from all other Republicans.
Instead, it may be that the preference were for any members of congress who voted in favour of the \texttt{easa} and \texttt{imm} bills. The point we are making has little material impact, but
is rather important for ones' intuition of what preference relations represent, or how they may be constructed. For the time being, we remain agnostic to this latter question---how they might
be constructed. This will be the subject of \Cref{subsubsection:finding-order}.

Preferential contexts allow us to speak about \textit{defeasible conditionals} in the setting of FCA. Much like classical implications in FCA, defeasible conditionals are defined over an attribute
set $M$, and so the expression $A \twiddle B$ should be understood to be saying that the ``typical'' objects which have all attributes from $A$ have all attributes from $B$. Defeasible conditionals
are satisfied with respect to a preferential context, admitting the following definition.

\begin{definition}
	\label{definition:satisfaction-preferential-context}

	Let $\pcontext = \PGMI$ be a preferential context, $A,B \subseteq M$ subsets of attributes, and $A \twiddle B$ a \emph{defeasible conditional}. We say that $A \twiddle B$ is \emph{satisfied}
	in $\pcontext$ if and only if $\Mind{A}\subseteq B^{\downarrow}$ and write $\pcontext \vDash_{\pcontext}A \twiddle B$.
\end{definition}

The condition for satisfaction of $A \twiddle B$ is that all the objects in the minimised-derivation of $A$ are members of the derivation of $B$. We might express this equivalently using the
minimal-return operator, giving us $B \subseteq \Minr{A}$.

\begin{example}
	In the preferential context composed from \Cref{figure:voting-records} and \Cref{figure:preference-relation-congress} the following holds:
	\begin{itemize}
		\item $\{\texttt{dem}\} \twiddle \{\texttt{crm}\}$

		\item $\{\texttt{dem,abr}\} \ntwiddle \{\texttt{crm}\}$
	\end{itemize}
\end{example}

With respect to the preference relation, members of congress who were Democrats usually voted in favour of the crime bill, while those Democrats who additionally voted to adopt the budget
resolution usually did not. It is clear that the logic is non-monotonic, and that it captures the spirit of KLM style reasoning in the sense that we are now able to speak about proto-typical
properties of classes. It might be suspected that in fac this corresponds completely to the preferential systems discussed earlier. We now show that this correspondence is not complete.

\begin{theorem}
	\label{theorem:preferential-context-soundness}

	The consequence relation $\twiddle_{\pcontext}$ induced by a preferential context $\PGMI$ satisfies \textit{Reflexivity, Left-logical equivalence, Right weakening, And,} and \textit{Cautious
	Monotony}.
\end{theorem}

\begin{proof}
	\label{proof:preferential-context-soundness}

	Let $\pcontext = \PGMI$ be a preferential context, and $A,B,C,D \subseteq M$ sets of attributes.

	For \textit{Reflexivity}, it needs to be shown that $\pcontext \vDash A \twiddle A$, equivalently that $\Mind{A}\subseteq A^{\downarrow}$. This holds by definition of the minimal-derivation
	operator. \lucas{Make sure this holds if $A^{\downarrow}= \emptyset$}

	For \textit{Left-logical equivalence}, if $\pcontext \vDash_{\pcontext}A \twiddle C$ and $A = B$, then $\pcontext \vDash_{\pcontext}B \twiddle C$. Assume that
	$\pcontext \vDash A \twiddle C$ and $A = B$ (by usual set equality). By this assumption, (i) $\Mind{A}\subseteq C^{\downarrow}$ and (ii) $\Mind{A}= \Mind{B}$ both hold. It is then clear that
	we can make the substitution that results in $\Mind{B}\subseteq C^{\downarrow}$, which is equivalent to $\pcontext \vDash B \twiddle C$.

	For \textit{Right weakening} we show that if $\pcontext \vDash A \rightarrow B, B \twiddle C$ then $\pcontext \vDash A \twiddle C$.
\end{proof}

Failure to satisfy the \textit{Or} postulate is due to limitations of FCA's attribute logic. In the essential setting, attribute logic is restricted to consideration of conjunctions of positive
attributes, without the expressivity for negation or disjunction. It is a frustrating point, since it is quite obvious what disjunction should mean in this setting.

\begin{definition}
	\label{definition:preferential-context-compound}
\end{definition}

\begin{theorem}
	\label{theorem:preferential-context-completeness}
\end{theorem}

\begin{proof}
	\label{proof:preferential-context-completeness}
\end{proof}
\begin{proposition}
	\label{proposition:preferential-context-irrational}
\end{proposition}
\begin{example}
	\label{example:counter-example-rationality}
\end{example}
\begin{definition}
	\label{definition:ranked-context}
\end{definition}
\textcolor{red}{a ranked context clarifies (removes equivalent objects)}
\subsubsection{Finding Order}
\label{subsubsection:finding-order}Thus far, the discussion on finding a suitable preference relation has been omitted. It is a non-trivial task to compare objects in a context.
\begin{algo}
	{\textsc{ObjectRank}}

	\label{algorithm:ObjectRank}

	\Require A set $\Delta$ of defeasible conditionals over an attribute set $M$ \Require A formal context $\GMI$

	\Ensure A ranking $\mathsf{R}\colon G \to \mathbb{N}$ such that $\Delta$ holds in $\RGMI$ if $\GMI$ is $\Delta$-compatible; $\bot$, otherwise.

	\State $i \coloneq 0$ \State \textit{initialise} $\mathsf{R}(g) \coloneq \vert G \vert$ for all $g \in G$ \State $\Gamma \coloneq \Delta$

	\While{$\exists \, g \in G \colon \mathsf{R}(g) = \vert G \vert$} \ForAll{$g \in G$ with $\mathsf{R}(g) = \vert G \vert$} \If{$g \vDash \Gamma$} \textit{update} $\mathsf{R}(g) = i$ \EndIf
	\If{$\forall \, g \in G \colon \mathsf{R}(g) \neq i$} \State \Return $\bot$ \EndIf \State $\Gamma \coloneq \{(\phi \twiddle \psi) \in \Gamma \mid g \nvDash \phi \; \text{for all}\; g \in G
	\; \text{with}\; \mathsf{R}(g) = i\}$ \State $i \coloneq i + 1$ \EndFor \EndWhile

	\State \Return $\mathsf{R}$
\end{algo}

\subsubsection{Complexity}
\section{Entailment}
\subsection{Contextual Rational Closure}
\begin{definition}
	A conditional $A \twiddle B$ \emph{rationally follows} from a set $\mathcal{T}$ of conditionals if and only if $\mathbb{T}, \mathcal{T}\dentails A \twiddle B$, where $\mathbb{T}$ is the \emph{test
	context}. It \emph{contextually follows} when $\mathbb{R}, \mathcal{T}\dentails A \rightarrow B$.
\end{definition}
\begin{definition}
	A set $\mathcal{T}$ is \emph{closed} with respect to a pair if and only if it contains every conditional that is rationally entailed. That is, if
	$\mathcal{T}= \{A \twiddle B \mid A,B \subseteq M \tand \mathbb{T},\mathcal{T}\dentails A \twiddle B\}$. $\mathcal{T}$ is \emph{contextually closed} if it contains every conditional that
	is contextually entailed, i.e. $\mathcal{T}= \{A \twiddle B \mid A,B \subseteq M \tand \mathbb{R}, \mathcal{T}\dentails A \rightarrow B\}$.
\end{definition}
\begin{definition}
	A set $\mathcal{T}$ of conditionals is \emph{rationally complete} with respect to $\mathbb{T},\Delta$ if and only if every conditional that is in the rational closure of
	$\mathbb{T}, \Delta$ rationally follows from $\mathcal{T}$, i.e., $\mathbb{T},\mathcal{T}$ and $\mathbb{T}, \Delta$ define the same entailment relation.
\end{definition}
Is it possible to find a$\mathcal{T}$such that$\mathcal{T}$is a compact form of$\Delta$. I.e., it should produce the same ranking on$\mathbb{T}$.
\begin{definition}
	A set $\mathcal{T}$ of conditionals is \emph{contextually complete} with respect to $\mathbb{R}, \Delta$ if and only if every implication that is in the contextual closure of
	$\mathbb{R}, \Delta$ rationally follows from $\mathcal{T}$.
\end{definition}
A set of conditionals that is contextually complete to a ranked context$\mathbb{R}$and constraint set$\Delta$provides abstracted perspective, in terms of defeasible conditionals, on the info
rmation contained in the pair$(\mathbb{R}, \Delta )$. It is now the goal, as was done in \Cref{subsubsection:implication-bases}, to find such a set that is non-redundant, or the \textit{rational
basis}of$(\mathbb{R},\Delta)$.

\begin{definition}
	\label{definition:defeasible-closure}

	Let $\mathcal{T}$ be a set of defeasible conditionals over $M$. The operator $X \mapsto \Cn{\twiddle}(X)$ given by $X \cup \{Y \subseteq M \mid X \twiddle Y\}$ is called the \emph{defeasible
	closure} of $X$. It is idempotent, extensive, and non-monotonic.
\end{definition}

\textcolor{red}{
\begin{remark}
	If $\mathcal{T}$ were the defeasible basis we would want $\Cn{\twiddle}(X) = (\underline{X^{\downarrow}})^{\uparrow}$
\end{remark}
}

\subsection{Rational Closure}

\subsection{A Basis for Ranked Contexts}

\subsection{Lexicographic Closure}

\begin{figure}
	\centering
	\begin{cxt}
		\cxtName{\textbf{\texttt{games}}}

		\atr{\texttt{single-p}}

		\atr{\texttt{multi-p}}

		\atr{\texttt{spatial-r}}

		\atr{\texttt{verbal-r}}

		\atr{\texttt{planning}}

		\atr{\texttt{motor-skills}}

		\atr{\texttt{competitive}}

		\atr{\texttt{cooperative}}

		\atr{\texttt{imaginative}}

		\atr{\texttt{constructive}}

		\atr{\texttt{modular}}

		\atr{\texttt{moving-parts}}

		\atr{\texttt{goal}}

		\atr{\texttt{rules}}

		\atr{\texttt{ends}}

		\obj{.xx.x.x.x..xxxx}{\texttt{chess}}

		\obj{x.x.x.x....xx.x}{\texttt{rubik's cube}}

		\obj{.xx.x..xxxxx...}{\texttt{lego}}

		\obj{x....xx.x..x..x}{\texttt{yo-yo}}

		\obj{.xx.xxx...xxxxx}{\texttt{pick-up-sticks}}

		\obj{.x.....xxx.x...}{\texttt{toy soldiers}}
	\end{cxt}
\end{figure}

% \begin{figure}
% 	\centering
% 	\begin{cxt}
% 		\cxtName{\textbf{\texttt{parks}}}

% 		\atr{\texttt{walking-paths}}

% 		\atr{\texttt{body-of-water}}

% 		\atr{\texttt{ball-games}}

% 		\atr{\texttt{allowed-on-grass}}

% 		\atr{\texttt{grass}}

% 		\atr{\texttt{bicycling}}

% 		\atr{\texttt{rowing}}

% 		\atr{\texttt{playground}}

% 		\atr{\texttt{public}}

% 		\atr{\texttt{private}}

% 		\atr{\texttt{allowed-picnics}}

% 		\atr{\texttt{animals}}

% 		\atr{\texttt{in-city}}

% 		\atr{\texttt{trees}}

% 		\obj{}{\texttt{Central Park}}

% 		\obj{x.x.x.x....xx.x}{\texttt{Kruger}}

% 		\obj{.xx.x..xxxxx...}{\texttt{Großer Garten}}

% 		\obj{x....xx.x..x..x}{\texttt{Jardin du Luxembourg}}

% 		\obj{.xx.xxx...xxxxx}{\texttt{Washington Mall}}

% 		\obj{.x.....xxx.x...}{\texttt{Tempelhof}}
% 	\end{cxt}
% \end{figure}