\chapter{Defeasible Reasoning}
\label{chapter:defeasible-reasoning}

Non-monotonicity in logical systems has been the focus of study for decades, and several distinct formalisms have been developed.
The motivation is to expand the inference power beyond that of the classical, to a more credulous one. This work is
primarily interested in non-monotonic reasoning following the preferential semantics introduced by Shoham
\cite{shohamSemanticApproach}, for which there is a well-developed model theory which further benefits from an obvious analogue
to notions central to FCA. In particular, we are interested in the framework put forward by Kraus, Lehmann, and Magidor \cite{kraus1990nonmonotonic,lehmann1992what},
frequently initialised to the `KLM framework'.

In the proceeding, we provide a background on non-monotonic reasoning in general, before introducing the KLM framework. We
assume all the same conventions introduced in \Cref{section:propositional-logic} for the technical discussions which occur
in this chapter.

\section{Background on Non-monotonic Reasoning}
\label{section:nmr-background} Near the end of the previous chapter, the matter of consequence was discussed in a very
formal sense. It is perhaps helpful to distinguish this subject---classical consequence---from the common concept, as
the former yields some surprising results which do not appear at all congruent with how a person, or otherwise intelligent
agent should reason \cite{tarski1936consequence, kraus1990nonmonotonic}. For a demonstration of a result which may be
\textit{surprising} in this way, consider the following propositions:
\vspace{1em}
\begin{enumerate}
	\item $\texttt{human}\rightarrow \texttt{chronological time}$

	\item $\texttt{soldier}\rightarrow \texttt{human}$

	\item $\texttt{billy pilgrim}\rightarrow \neg \texttt{chronological time}$
\end{enumerate}
\vspace{1em}

Knowing these propositions, if we were to encounter an individual in fatigues we might find it sensible---by propositions
2 and 1---to infer that the individual experienced time chronologically. If we were to later learn that the individual
was, in fact, Billy Pilgrim, given proposition 3, we should like to retract our prior inference, replacing it with the knowledge
that the individual does not experience time chronologically.

Such recourse is not, as it stands, possible: When we see the that the individual is a soldier, the possible worlds satisfying
our knowledge are reduced to the single model: $\{\overline{b},c,h,s\}$ (where $b$: \texttt{billy pilgrim}, $c$: \texttt{chronological
time}, $h$: \texttt{human}, and $s$: \texttt{soldier}). Should we later learn that the individual is indeed Billy
Pilgrim, the theory becomes inconsistent.
%TODO: Add section in preliminaries about principle of explosion
And, by the principle of explosion, discussed in \Cref{subsection:logical-consequence}, our theory now entails that the individual
experiences time both chronologically and non-chronologically; indeed our theory entails everything and is accordingly
worthless.

This property of classical logic---that adding new information never results in retraction of pre-existing knowledge---is
called monotonicity. Monotonicity requires that when we make a claim like \say{humans experience time chronologically}, we
must be absolutely sure of ourselves, so as to never worry about needing to retract an inference. This is, of course, too
strict a requirement as we cannot determine for all future, present, and past humans if it were the case that they experienced
time chronologically. If we remain in the classical realm, it seems our only options are to abandon our original claim or
risk explosion.

At this point it is a good idea to provide some clarification on how we might begin to approach this issue. Continuing
with the same example---and continuing to allow ourselves to entertain the possibility of experiencing non-chronological
time---it would certainly be agreed that typically soldiers are human, and also that typically humans experience time chronologically.
To resolve that Billy Pilgrim is a soldier, and therefore a human, who does not experience chronological time, we need only
to point out that he is an atypical human.

To make the previous paragraph more formal, we remind the reader of the discussion held around
\Cref{definition:logical-consequence}: A formula $\phi$ is a logical consequence of a set $\Gamma$ thereof if every model
of $\Gamma$ is also a model of $\phi$. Put differently, there is no valuation (or, possible world) where $\Gamma$ is
true and $\phi$ is false. It follows directly that $\phi$ remains a logical consequence of $\Gamma \cup \{\psi\}$, since
$\Gamma$ is true in any world where $\Gamma \cup \{\psi\}$ is true.

It was pointed out by Shoham \cite{shohamSemanticApproach} that we may \say{bend the rules} and restrict semantic consideration
to a privileged subset of models deemed ``preferable''. We call these selected models the ``minimal'' models---a choice
that will become clearer as this chapter progresses.

% \section{Preferential Reasoning}
% \label{section:preferential-reasoning}
% In this work we are interested specifically in the KLM-style of non-monotonic reasoning, which will be introduced in the proceeding chapter. As a precursor, it is beneficial to discuss the semantic approach to non-monotonic reasoning put forward by Shoham \cite{shohamSemanticApproach}.

% \begin{definition}
% \label{definition:preferential-satisfaction-from-valuation}
% Given a formula $\phi \in \mathcal{L}$ and a valuation $u \in \mathcal{U}$, then $u$ \emph{preferentially satisfies} $\phi$ (and we write $u \vDash_\sqsubseteq \phi $) if and only if there is no other interpretation $v \in \mathcal{U}$ where $v \vDash \phi$ and $v \sqsubset u$. Then $u$ is a \emph{preferred}, or \emph{minimal} model of $\phi$.
% \end{definition}

% \begin{definition}
% \label{definition:preferential-satisfiability}
% A formula $\phi \in \mathcal{L}$ is \emph{preferentially satisfiable} if it has a preferred model.
% \end{definition}

% From the two definitions above it may be obvious that every preferentially satisfiable formula is indeed classically satisfiable, while the converse is not true: To see why we need only consider an infinitely descending chain of valuations, and whether a tautology is preferentially satisfiable in such a structure.
\section{The KLM Framework}
The KLM framework for non-monotonic reasoning was initially described by a collection of consequence relations satisfying
certain axioms---frequently called the \textit{rationality postulates}---with each successive system being stronger than
its predecessor. We borrow a nice story from Dov Gabbay \cite{gabbay1985theoreticalFoundations} which motivates why consequence
relations are a good starting point for the study of a non-monotonic system.

Paraphrasing, he begins by asking the reader to imagine a machine that does non-monotonic inference in some domain. The machine
represents knowledge as formulae and so we pose queries of the form \say{Does $\psi$ non-monotonically follow from $\phi$?}.
Something goes awry (suppose some coffee was spilled), calling into question whether the logic of the machine still
functions correctly. Even worse, the interface, which tells us what real-world instance each formula maps to, is
destroyed and so function cannot be evaluated based on the meaning of the formulae the machine reasons on. How might we then
evaluate the machine's function?

If we were interested in classical consequence, we would be well-equipped to assess the correctness of the machine by
determining if it satisfied reflexivity, monotonicity, and cut (we point to \Cref{definition:consequence-relations} as a
reminder). This is precisely the starting point that Kraus, Lehmann, and Magidor took up in \cite{kraus1990nonmonotonic},
suggesting that before getting to the semantics of a non-monotonic system, it is a good idea to formalise axiomatise the
system as a consequence relation satisfying certain properties.

The rationality postulates are precisely this axiomatisation, characterising a sensible pattern of reasoning for non-monotonic
systems. We use `$\twiddle$' (pronounced ``twiddle'') instead of `$\vdash$' to denote a non-monotonic consequence relation.
As we may expect, $\phi \twiddle \psi$ has the same meaning as $(\phi, \psi) \in \; \twiddle$, and $\phi \ntwiddle \psi$
as $(\phi, \psi)\not \in \; \twiddle$. We may, at times of potential confusion, use a subscript to disambiguate which consequence
relation is being referred to, and so $\twiddle_{C}$ would refer to a cumulative relation, as defined below.

% An expression like $\phi \twiddle \psi$ is to be understood as saying \say{If $\phi$ holds, then $\psi$ is a typical consequence}, we call
% expressions of this nature \textit{conditional assertions}. The properties that a particular consequence relation satisfies are represented
% as rules of inference in the style of conditional assertions.

\subsection{Cumulative Reasoning}
\label{subsection:system-c} \index{non-monotonic reasoning! system C}

\textit{Cumulative consequence relations}, otherwise referred to as \textit{System C}, represent the weakest of the systems
in the KLM framework. We follow the same exposition, from weaker to stronger systems, as \cite{kraus1990nonmonotonic}: this
approach will minimise repetition, as each successive system inherits all properties of its predecessor.

\begin{definition}
	\label{definition:cumulative-consequence-relation} A consequence relation $\twiddle$ is a \emph{cumulative consequence
	relation} if and only if it satisfies the properties of \emph{Reflexivity, Left Logical Equivalence, Right Weakening, Cut}
	and \emph{Cumulative Monotonicity}.
\end{definition}

The first axiom, \textit{Reflexivity}, is largely self justifying: It makes little sense to speak about a notion of consequence
that does not satisfy this property.

\begin{align}
	\label{postulate:ref}\inferLeft{Reflexivity}{}{\phi \twiddle \phi}
\end{align}

The justification for \textit{Left Logical Equivalence} is a bit more opaque; the principle is that if two scenarios
represent the same state of affairs, and in one of these scenarios it we typically expect some consequence, then we
should expect the same in the other scenario.

\begin{align}
	\label{postulate:lle}\inferLeft{Left Logical Equivalence}{\vdash \phi \leftrightarrow \psi, \quad \phi \twiddle \gamma}{\psi \twiddle \gamma}
\end{align}

\textit{Right Weakening} allows the preservation of classical consequence within the logic. It says that if it is always
the case that from knowing $\psi$ we conclude $\gamma$, and from knowing $\phi$ we normally conclude $\psi$, then we are
entitled to hold the view that from $\phi$ we normally expect $\gamma$ as well.

\begin{align}
	\label{postulate:rw}\inferLeft{Right Weakning}{\vdash \psi \rightarrow \gamma, \quad \phi \twiddle \psi}{\phi \twiddle \gamma}
\end{align}

\textit{Cautious Monotony} (which has also be called \textit{Cumulative Monotony} by Makinson \cite{makinson2003bridges},
and \textit{Restricted Monotony} by Gabbay \cite{gabbay1985theoreticalFoundations}) corresponds to the notion that if we
are in an epistemic state $\phi$ where one expectation, among others, is that $\psi$ holds. Learning that $\psi$ indeed holds
should not alter the epistemic state in such a way that the \textit{other} expectations are abandoned, and so the new state,
$\phi \land \psi$, we should expect everything that was expected when all we knew was $\phi$. In other words, we reason monotonically
with respect to expected information.

\begin{align}
	\label{postulate:cm}\inferLeft{Cautious Monotony}{\phi \twiddle \psi, \quad \phi \twiddle \gamma }{\phi \land \psi \twiddle \gamma}
\end{align}

Certain other rules may derived from the presence of already discussed postulates. A version of \textit{Cut}

\begin{align}
	\label{postulate:cut}\inferLeft{Cut}{\phi \land \psi \twiddle \gamma, \quad \phi \twiddle \psi}{\phi \twiddle \gamma}
\end{align}

The original version due to Gentzen \cite{Ben1993Mathematical} is presented as:
\begin{align}
	\inferLeft{Monotonic Cut}{\phi \land \psi \twiddle \gamma, \quad \alpha \twiddle \psi}{\phi \land \alpha \twiddle \gamma}
\end{align}
implies monotonicity, as it requires that if $\psi$ is a typical consequence of $\alpha$, then it must remain a consequence
of $\alpha \land \phi$: ergo, monotonicity. The former variation does not enforce this, and rather says \say{Suppose I have certain knowledge of $\phi$, and that if I were to assume $\psi$ I should expect to conclude $\gamma$. Then if I can show that infact $\psi$ was already an expected consequence of knowing $\phi$, I should expect that $\gamma$ follows from $\phi$}.
When considered alongside the argument for Cautious Monotony, Cut seems obviously acceptable.

The \textit{And} postulate suggests that if $\psi$ and $\gamma$ are both expected consequence of $\phi$, then their
conjunction is also expected. This postulate fails in probabilistic systems, such as \textit{association rules} \cite{gabbay1985theoreticalFoundations}.
\begin{align}
	\label{postulate:and}\inferLeft{And}{\phi \twiddle \psi, \quad \phi \twiddle \gamma}{\phi \twiddle \psi \land \gamma}
\end{align}

The following Lemma is a helpful intuition pump, it is largely why the term \textit{cumulative} is used for this system:
it suggests that we can use the consequences of plausible inferences to make further plausible inferences.
\begin{lemma}
	\label{lemma:cut-cautious} We can cover the properties of \emph{Cut} and \emph{Cumulative Monotonicity} with the following
	principle: \say{If $\phi \twiddle \psi$, then the typical consequences of $\phi$ and $\phi \land \psi$ coincide}.
\end{lemma}

Certain, frequently discussed properties in classical logic do not hold in cumulative consequence relations. Most
obviously, cumulative consequence relations do not satisfy
\begin{align}
	\label{postulate:monotonicity}\inferLeft{Monotonicity}{\psi \rightarrow \gamma, \quad \gamma \twiddle \phi}{\psi \twiddle \phi}
\end{align}
In addition, \textit{Transitivity} and \textit{Contraposition} which both imply monotonicity when considered alongside
the other rules of cumulative relations
\begin{align}
	\label{postulate:trans}\inferLeft{Transitivity}{\phi \twiddle \psi, \quad \psi \twiddle \gamma}{\phi \twiddle \gamma}
\end{align}
Transitivity, in \Cref{proof:transitivity}, was shown to be quite useful as a derived rule of a Hilbert system; but, for
our purposes, it will not do. Consider the example at the beginning of this chapter, where the topic of whether we should
infer that Billy Pilgrim experiences chronological time. Transitivity requires that we infer he does: Billy Pilgrim is a
soldier, soldiers are human, and humans experience time chronologically. But this is precisely the inference we do not
want, and so transitivity must be abandoned.
%
\begin{align}
	\label{postulate:contra}\inferLeft{Contraposition}{\phi \twiddle \psi}{\neg \psi \twiddle \neg \phi}
\end{align}

These properties are undesirable as they imply monotonicity, precisely what we are interested in avoiding. However, it
is worth reminding ourselves of the discussion at the beginning of this chapter: That our aim is to develop a system which
allows for more (credulous) inferences to be made. A question that arises is whether non-monotonic systems should be
\textit{supraclassical} \cite{makinson2003bridges}: should classical inferences be preserved? Under the framing we have adopted,
that classical deductions require iron-clad proofs beyond which is often practical, and that these requirements could be
relaxed in order to make more useful (but retractable) inferences. Then, if we have a classical proof of something it should
of course hold in a non-monotonic system.

\begin{align}
	\label{postulate:supraclassical}\inferLeft{Supraclassical}{\phi \rightarrow \psi}{\phi \twiddle \psi}
\end{align}

\textcolor{red}{Diagram ovals of worlds}

We will skip over any discussion of semantics for cumulative consequence relations, and rather opt to introduce these in
the next section where we discuss preferential consequence relations.

\subsection{Preferential Reasoning}
\label{subsection:system-P} \index{non-monotonic reasoning! system P}

We, quite quickly, move on from cumulative to \textit{preferential consequence relations}, or \textit{system P}. The
reason being that system P is strictly stronger than system C, and includes in its' axiomatisation something analogous
to the more significant part of the deduction theorem, as well as disjunction. The semantics of system P, to be outlined
in \Cref{subsubsection:preferential-interpretations}, are similar to the approach which was proposed by Shoham in
\cite{shohamSemanticApproach}.

Fortunately, the definition of a preferential consequence relation is almost identical to that of cumulative one, with
only the addition of the \textit{Or} postulate.

\begin{definition}
	\label{definition:preferential-relation}

	The consequence relation $\twiddle$ is a \emph{preferential consequence relation} if and only if it satisfies the properties
	of \emph{Reflexivity, Left Logical Equivalence, Right Weakening, Cut, Or,} and \emph{Cumulative Monotonicity}.
\end{definition}

As a justification for \textit{Or}, consider that \textit{If Billy were abducted by aliens, normally he would be traumatised},
but also \textit{If Billy witnessed the fire-bombing of Dresden, normally he would be traumatised}. If we know that at
least one of these events happened we should be allowed to conclude that Billy were traumatised, since either of them would
normally allow this inference.

\begin{align}
	\inferLeft{Or}{\phi \twiddle \gamma, \quad \psi \twiddle \gamma}{\phi \lor \psi \twiddle \gamma}
\end{align}

The addition of \textit{Or} to the rest of the properties of system C allow for useful derived rules. For instance, \textit{S}

\begin{align}
	\inferLeft{S}{\phi \land \psi \twiddle \gamma}{\phi \twiddle \psi \rightarrow \gamma}
\end{align}

The derived rule \textit{S} is analogous to the deduction theorem (cf. \Cref{axiom:deduction-theorem}), and

\begin{lemma}[\cite{shohamSemanticApproach}]
	For $\phi, \psi \in \mathcal{L}$ and some valuation $u \in \mathcal{U}$, if $u \vdash \psi$ and $u \twiddle \phi$, then
	$u \twiddle \phi \land \psi$.
\end{lemma}

\subsubsection{Preferential Interpretations}
\label{subsubsection:preferential-interpretations}

The semantics of system P provided by Kraus, Lehmann, and Magidor \cite{kraus1990nonmonotonic} are based on the \textit{preference
logics} proposed by Shoham \cite{shohamSemanticApproach}. The fundamental idea is that valuations, or \textit{worlds}, can
be ordered by a \textit{preference relation}, so that one world being preferred to another is a normative claim that we should
consider deductions which hold in the preferred world---but may not in other worlds---plausible.

It will be useful to recall some definitions from propositional logic: a valuation $u \in \mathcal{U}$ is a \textit{model}
of a formula $\phi \in \mathcal{L}$ if it satisfies $\phi$, $\phi$ is \textit{satisfiable} if it has a model. Another
formula $\psi$ is a \textit{logical consequence} of $\phi$ if the models of $\phi$ are a subset of the models of $\psi$.

We now introduce analogous definitions for these ideas in preferential logic. The first change is that we consider a
richer language, $\Lang$, which is simply $\lang$ extended with the new connective `$\twiddle$' such that
$\phi \twiddle \psi$ is a sentence in $\Lang$ when $\phi, \psi \in \mathcal{L}$. By this description, nested statements
of the form `$\phi \twiddle (\psi \twiddle \gamma )$' are not permitted. At times, we may wish to distinguish between
formulae in $\Lang$ which do not use `$\twiddle$', and so we refer to these as classical, and the alternative as
defeasible formulae.

\begin{definition}
	\label{definition:preferential-interpretation} \index{non-monotonic reasoning! system P! preferential interpretation}

	A \emph{preferential interpretation} $\Pin$ is a triple where $S$ is a set of \emph{states}, $l: S \to \mathcal{U}$ is
	a function mapping states to valuations, and $\prec$ is a strict partial-order on $S$.
\end{definition}

We can then define the notion of satisfaction, relative to a preferential interpretation, as:

\begin{definition}
	\label{definition:state-satisfaction}

	Given a preferential interpretation $\pin$, we say that a state $s \in S$ \emph{satisfies} a classical formula
	$\phi \in \Lang$ if and only if the valuation $l(s) \vDash \phi$. In this case, we write $\pin, s \vDash \phi$, and use
	$\hat{\phi}$ to denote the set $\{s \in S \mid \pin,s \vDash \phi \}$. Then, $\pin$ satisfies $\phi$ if and only if $\pin
	, s \vDash \phi$ for all $s \in S$.
\end{definition}

It is clear from this definition that a classical formula which is satisfiable in a preferential interpretation will
always be satisfiable (i.e., in the classical sense, it will have a model). The converse, however, is not guaranteed
under this definition. We suggest another reading of satisfaction in a preferential interpretation, which illuminates
this point: a classical formula $\phi$ is satisfied by a preferential interpretation $\Pin$ if and only if there is some
state $s \in S$ such that $l(s) \vDash \phi$ and there is no other $s'\in S$ where $s' \prec s$.

So, a classical formula is satisfiable in a preferential model so long as it has a model, and there is no infinitely descending
chain of models in the preference relation. Consequently, it is often required that the preference relation satisfy the following
condition,

\begin{definition}
	\label{definition:smoothness} \index{binary relation! smoothness}

	A preferential interpretation $\Pin$ satisfies the \textit{smoothness condition} if and only if each non-empty subset $T
	\subseteq S$ has a $\prec$-minimal state.
\end{definition}

\textcolor{red}{Maybe this is too strong, we need only that every $\phi \in \Lang$ have a set of models $\hat{\phi}$ that
satisfy the smoothness condition}

A preferential interpretation that satisfies the smoothness condition is called \textit{well-founded}, or \textit{bounded}
\cite{shohamSemanticApproach,lehmann1992what}. Of course, every preferential interpretation where the set of states is finite
is well-founded.

As a matter of intuition, the enforcing the smoothness condition on a preference relation seems completely agreeable: It
makes little sense, when the express aim is to restrict consideration to a subset of the most preferred worlds, to order
an infinite collection of worlds in terms of how preferreable they are, and not have a starting point describing the
\say{most preferable} worlds.

We then remark that the, under the assumption of well-foundeness, a (classical) formula is satisfiable in a preferential
interpretation if and only if it is classically satisfiable (i.e., it has a model). We are, however, more interested in
satisfaction of defeasible statements, we first define what it means to restrict consideration to preferred worlds:

\begin{definition}
	\label{definition:state-minimal}

	Given a preferential interpretation $\Pin$ and some $\phi \in \Lang$, we write $\underline{\hat{\phi}}$ to denote the set
	$\{s \in \hat{\phi}\mid \nexists s' \in \hat{\phi}\text{ such that }s' \prec s \}$, which is the set of \textit{preferred
	states} which satisfy $\phi$. We may also call this the set of \textit{minimal states} which satisfy $\phi$.
\end{definition}

Then,

\begin{definition}
	\label{definition:preferentially-satisfiable}

	A defeasible formula $\phi \twiddle \psi \in \Lang$ is \emph{satisfied} by a preferential interpretation $\Pin$ if and
	only if the every minimal state $s \in \underline{\hat{\phi}}$ which satisfies $\phi$ also satisfies $\psi$. In this
	case we write $\pin \VDash \phi \twiddle \psi$ and say that $\pin$ is a \emph{preferential model} of $\phi \twiddle \psi$.
\end{definition}

\begin{example}
	\label{example:preferential-interpretation}

	Consider the knowledge base
	\[
		\Delta = \big\{\;h \twiddle c, \; s \twiddle h, \; b \rightarrow \neg c \; \big \}
	\]

	containing both defeasible conditionals, as well as classical propositional formulae. The knowledge base encodes the setting
	where \say{Humans normally experience time chronologically}, \say{Soldiers are normally human}, and \say{Billy Pilgrim experiences non-chronological time.}

	There are many preferential interpretations which may satisfy $\Delta$, one of these is depicted below: Each set contains
	the states which map to a valuation that satisfies the respective propositional atom. The shaded area of a set
	represents the minimal states for that set, and so the green shading represents the set $\hat{s}$.

	\begin{figure}[H]
		\centering
		\begin{tikzpicture}
			% outer frame
			% \draw[help lines] (0,0) grid (8,6);
			\draw[rounded corners=5pt] (0,0) rectangle (8,6);

			% translate all three shapes at once
			\begin{scope}[shift={(-0.25,0)}]
				% shape 1: HUMANS
				% \filldraw[rounded corners=5pt, fill=blue, fill opacity=0] (1,6) rectangle (5,3);
				\node (human) at (0.75,5.25) [] {$h$};
				\draw[rounded corners=5pt] (1,6) -- (1,3) -- (5,3) -- (5,6);

				% % shape 2: CHRONOLOGICAL
				\filldraw[rounded corners=5pt, fill=red, fill opacity=0]
					(1.5,4) --
					(3,4) --
					(3,2) --
					(7.5,2) --
					(7.5,0.5) --
					(1.5,0.5) --
					cycle;
				%   (1.5,3.25) --
				%   (3,3.25) --
				%   (3,1.75) --
				%   (6.5,1.75) --
				%   (6.5,0.25) --
				%   (1.5,0.25) --
				%   cycle;
				\node (chronological) at (1.25,2.25) [] {$c$};

				% % shape 3: SOLDIER
				\filldraw[rounded corners=5pt, fill=green, fill opacity=0] (4,4.5) -- (7,4.5) -- (7,1.5) -- (4,1.5) -- cycle;
				%   (3.75,3) --
				%   (3.75,4.5) --
				%   (8,4.5) --
				%   (8,1) --
				%   (5.5,1) --
				%   (5.5,3) --
				%   cycle;
				\node (soldier) at (7.25,4) [] {$s$};

				% % shape 4: BILLY
				\filldraw[rounded corners=5pt, fill=green, fill opacity=0]
					(3.25,5.25) --
					(4.5,5.25) --
					(4.5,3.75) --
					(3.25,3.75) --
					cycle;
				\node (billy) at (3,5) [] {$b$};

				%  Minimal Human
				\begin{scope}
					\clip[rounded corners=0pt] (1.78,4) rectangle (2.45,3.25);
					\fill[pattern=dots, pattern color=blue] (0,0) rectangle (10,6);
				\end{scope}

				%  Minimal Soldier
				\begin{scope}
					\clip[rounded corners=0pt] (4.25,4.25) rectangle (4.75,3.5);
					\fill[pattern=dots, pattern color=green] (0,0) rectangle (10,6);
				\end{scope}

				%  Minimal Billy
				\begin{scope}
					\clip[rounded corners=5pt] (4,4.5) rectangle (4.5,3.75);
					\fill[pattern=north east lines, pattern color=red] (0,0) rectangle (10,6);
				\end{scope}
			\end{scope}
		\end{tikzpicture}
		\caption{A preferential interpretation}
		\label{figure:preferential-interpretation}
	\end{figure}
	It is quite obvious to spot an illustration of \Cref{lemma:cut-cautious}, $b \twiddle s$ and so all the typical
	consequences of $b$ and $b \land s$ coincide. This clearly holds, as they are precisely the same set of states.
\end{example}
\begin{theorem}[Soundness]
	\label{theorem:soundness-preferential}

	If $W$ is a preferential interpretation which defines the consequence relation $\twiddle_{W}$, then $\twiddle_{W}$ satisfies
	\textit{Reflexivity, Left Logical Equivalence, Right Weakening, Or,} and \textit{Cautious Monotony}.
\end{theorem}

\begin{theorem}[Completeness]
	\label{theorem:completeness-preferential}

	If $\twiddle_{P}$ is a preferential consequence relation, then there exists a preferential interpretation $W$ which induces
	the consequence relation $\twiddle_{W}$ such that $\twiddle_{P}$ is precisely $\twiddle_{W}$.
\end{theorem}

The decision to construct a preference relation on states rather than valuations directly, and then map states to
valuations, may seem arbitrary, and it would be ideal to do away with this layer of obfuscation. Certainly, Shoham
\cite{shohamSemanticApproach} makes no such distinction. Kraus et al. \cite{kraus1990nonmonotonic} make the point that
there are preferential consequence relations with corresponding preferential interpretations which contain duplicate
states, which cannot be represented by another preferential interpretation without these duplicates.

\begin{example}
	Consider the more fine-grained depiction of the preferential interpretation in \Cref{figure:preferential-interpretation}
	in the style of a Hasse diagram:
\end{example}




\subsection{Preferential Entailment}

RE iT not being nmr We are willing to accept these credulous inferences precisely because they may be retracted we may withdraw
previous inferences under learning new facts.

\section{Rational Closure}

\subsection{Rational Consequence Relations}