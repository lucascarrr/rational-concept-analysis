\chapter{Defeasible Reasoning in Formal Concept Analysis}
\label{chapter:defeasible-reasoning-in-fca}

The conclusion of \Cref{part:1} marks the end of the scaffolding component of this work. The remaining chapters are devoted to the novel contributions of this work: the unification
of the ideas presented in \Cref{chapter:defeasible-reasoning,chapter:formal-concept-analysis}. That is, to develop an approach to defeasible reasoning in FCA in the style of
rational consequence.

Fortunately, the context switch is quite natural. As we progress, the reasons behind the selection of the KLM framework as an ideal candidate for defeasible reasoning in FCA should
become apparent. Indeed, understanding the proceeding chapter(s) is benefitted by frequently ``pattern-matching'' between the definitions presented here, and their counterparts from
prior chapters. So we encourage---and will attempt to facilitate---these frequent backwards-facing references.

\section{Motivation}
\label{section:motivation}

There are multiple views one may consider when thinking about non-monotonic reasoning in FCA, each with its own implicit justification for violating monotonicity. The most familiar
one, which has already been discussed in \Cref{chapter:defeasible-reasoning}, takes up the position that \textit{usually} there are exceptional and typical instances, and that there
should be a clear way to reason prototypically. Certainly, there is good reason for this argument in FCA; but this will be the subject of the next chapter.

The perspective we consider here views FCA as a tool which facilitates the extraction of information from data. Of course, data is frequently polluted with errors or noise, which inhibit
the discovery of useful information. It may be infeasible to try and correct or denoise data, and yet we hope that useful information can still be learned from it. Consider the
formal context below, which represents some of the United States congressional voting data for the House of Representatives, captured in 1984. The data only captures when a member
of congress voted in favour of a particular bill, whereas in reality members of congress could also reject the bill, or abstain from voting.

In addition, certain voting patterns might exist which are fail to be captured by attribute implications due to a member of congress crossing party lines.

% \textit{Exceptions} in this case are erroneous datum \cite{Sacco2024}. The point, then, is to facilitate reasoning from

\textcolor{red}{Represents a kind of counter factual: If these implications had been true, then these would follow}

\subsection{Association Rules}
\label{subsection:association-rules}

Concerns about the brittleness of---as well as the benefits of relaxing---the semantics of (classical) attribute implications are well known in the FCA literature, and in data
mining more generally \cite{Lakhal2005}. \textit{Association rules} are able express implication-like statements despite the existence of counterexamples in the data, or in our case
a context. These are statements like \say{$66\%$ of the members of congress who were \textit{Democrats}, who make up $50\%$ of congress, voted on the \textit{MX-Missile} bill}.

The \emph{support} of a set of attributes $X \subseteq M$ in a context $\GMI$ is the fraction of objects which have all attributes $X$ over the total number of objects; and so $\mathrm{supp}
(X) = \nicefrac{|X^{\downarrow}|}{|G|}$, the (relative) support of a rule $A \Rightarrow B$ is simply the support of $A$. The \textit{confidence} of a rule is a measure of how
frequently the premise and conclusion occur, relative to how frequently the premise occurs, formally
$\mathrm{conf}(A \Rightarrow B) = \nicefrac{\mathrm{supp}(X \cup Y)}{\mathrm{supp}(X)}$. In the example above, the support of the rule is $50\%$ and its' confidence is $66\%$.
Association rules which meet specified lower-thresholds for support and confidence are then accepted.

\begin{figure}[H]
	\centering
	\begin{cxt}
		\label{cxt:voting}

		\cxtName{\textbf{\texttt{Congressional}}}

		\atr{\texttt{dem}}

		\atr{\texttt{rep}}

		\atr{\texttt{inf}}

		\atr{\texttt{wpcs}}

		\atr{\texttt{abr}}

		\atr{\texttt{pff}}

		\atr{\texttt{esa}}

		\atr{\texttt{rgs}}

		\atr{\texttt{ast}}

		\atr{\texttt{aid}}

		\atr{\texttt{mxm}}

		\atr{\texttt{imm}}

		\atr{\texttt{scc}}

		\atr{\texttt{edu}}

		\atr{\texttt{srs}}

		\atr{\texttt{crm}}

		\atr{\texttt{dfe}}

		\atr{\texttt{easa}}

		\obj{.X.X.XXX...X.XXX.X}{\texttt{moc-0}}

		\obj{.X.X.XXX.....XXX..}{\texttt{moc-1}}

		\obj{X..X.XXX.......XXX}{\texttt{moc-6}}

		\obj{X.XXX...XXX.....XX}{\texttt{moc-34}}

		\obj{ X.XXX...XXX....... }{\texttt{moc-62}}

		\obj{ .X...XX....X.XXX.. }{\texttt{moc-79}}
	\end{cxt}
	\caption{A portion of the 1984 United States Congressional Voting Records dataset}
	\label{figure:voting-records}
\end{figure}

Indeed association rules offer a reasonable response the problems discussed at the beginning of this section. Suppose the thresholds for $\mathrm{MinSupp}$ and $\mathrm{MinConf}$
were fixed at $0.2$ and $0.5$, respectively. We would then accept the association rules:
%
\begin{align*}
	\texttt{dem}\Rightarrow \texttt{mxm}\quad \text{with}\; \mathrm{Sup}= 0.5 \; \text{and}\; \mathrm{Conf}= 0.66 \\
	\texttt{dem}\Rightarrow \texttt{dfe}\quad \text{with}\; \mathrm{Sup}= 0.5 \; \text{and}\; \mathrm{Conf}= 0.66
\end{align*}
%
Yet, the rule $\texttt{dem}\Rightarrow \texttt{mxm, dfe}$ is not accepted.

\section{Preferential Contexts}

\begin{definition}
	\label{definition:preferential-context} \index{formal context! preferential context}

	A \emph{preferential context} $\pcontext = \PGMI$ is a formal context $\GMI$ where $\prec$ denotes a strict partial order on the set of objects.
\end{definition}

Preferential contexts, albeit with a different motivation, were first introduced by Obiedkov \shortcite{Obiedkov2012}, and later on with our purposes in mind in \cite{Carr2024}. In
analogy to the preferential \textit{interpretations} defined earlier, the relation on objects conveys that for two objects $g,h \in G$, $g \preceq h$ should be interpreted as
saying that \say{Object $g$ is more preferable to $h$}. While preferential interpretations were defined by a relation on states, and a not-necessarily-injective mapping from states
to valuations, preferential contexts do not require states, and can construct the relation on objects directly. In essence, two distinct objects may have the same intent,
implicitly functioning like states.

Under the preference relation, we describe the $\prec$-minimal states through the process of \textit{minimisation}.
\begin{definition}
	\label{definition:minimisation}

	In a preferential context $\pcontext = \PGMI$, the \emph{minimisation} of a set $A \subseteq G$ is defined by
	\[
		\underline{A}\coloneqq \{\, g \in A \mid \nexists h \in A : (h \prec g) \,\}.
	\]
\end{definition}
%
Preferential contexts allow us to speak about \textit{defeasible conditionals} in the setting of FCA. Much like classical implications in FCA, defeasible conditionals are defined
over an attribute set $M$, and so the expression $A \twiddle B$ should be understood to be saying that the ``most preferred'' objects in a context that have all attributes from $A$
also have all attributes from $B$.

\begin{definition}
	\label{definition:satisfaction-preferential-context}

	Given a preferential context $\pcontext = \PGMI$ and $A,B \subseteq M$, then we say that $A \twiddle B$ is \emph{satisfied} by $\pcontext$ if and only if
	$\underline{A^\downarrow}\subseteq B^{\downarrow}$, and we write $\pcontext \vDash A \twiddle B$.
\end{definition}

\begin{theorem}
	\label{theorem:preferential-context-soundness}

	The consequence relation $\twiddle_{\pcontext}$ induced by a preferential context $\PGMI$ is \textit{preferential}.
\end{theorem}

\begin{proof}
	\label{proof:preferential-context-soundness}
\end{proof}

\begin{theorem}
	\label{theorem:preferential-context-completeness}
\end{theorem}

\begin{proof}
	\label{proof:preferential-context-completeness}
\end{proof}

\begin{proposition}
	\label{proposition:preferential-context-irrational}
\end{proposition}

\begin{example}
	\label{example:counter-example-rationality}
\end{example}

\begin{definition}
	\label{definition:ranked-context}
\end{definition}

\textcolor{red}{a ranked context clarifies (removes equivalent objects)}

\subsubsection{Finding Order}
\label{subsubsection:finding-order}

Thus far, the discussion on finding a suitable preference relation has been omitted. It is a non-trivial task to compare objects in a context.

\begin{algo}
	{\textsc{ObjectRank}}

	\label{algorithm:ObjectRank}

	\Require A set $\Delta$ of defeasible conditionals over an attribute set $M$ \Require A formal context $\GMI$

	\Ensure A ranking $\mathsf{R}\colon G \to \mathbb{N}$ such that $\Delta$ holds in $\RGMI$ if $\GMI$ is $\Delta$-compatible; $\bot$, otherwise.

	\State $i \coloneq 0$ \State \textit{initialise} $\mathsf{R}(g) \coloneq \vert G \vert$ for all $g \in G$ \State $\Gamma \coloneq \Delta$

	\While{$\exists \, g \in G \colon \mathsf{R}(g) = \vert G \vert$} \ForAll{$g \in G$ with $\mathsf{R}(g) = \vert G \vert$} \If{$g \vDash \Gamma$} \textit{update}
	$\mathsf{R}(g) = i$ \EndIf \If{$\forall \, g \in G \colon \mathsf{R}(g) \neq i$} \State \Return $\bot$ \EndIf \State $\Gamma \coloneq \{(\phi \twiddle \psi) \in \Gamma \mid g \nvDash
	\phi \; \text{for all}\; g \in G \; \text{with}\; \mathsf{R}(g) = i\}$ \State $i \coloneq i + 1$ \EndFor \EndWhile

	\State \Return $\mathsf{R}$
\end{algo}

\subsubsection{Complexity}

\section{Entailment}

\subsection{Contextual Rational Closure}

\begin{definition}
	A conditional $A \twiddle B$ \emph{rationally follows} from a set $\mathcal{T}$ of conditionals if and only if $\mathbb{T}, \mathcal{T}\dentails A \twiddle B$, where $\mathbb{T}$
	is the \emph{test context}. It \emph{contextually follows} when $\mathbb{R}, \mathcal{T}\dentails A \rightarrow B$.
\end{definition}

\begin{definition}
	A set $\mathcal{T}$ is \emph{closed} with respect to a pair if and only if it contains every conditional that is rationally entailed. That is, if
	$\mathcal{T}= \{A \twiddle B \mid A,B \subseteq M \tand \mathbb{T},\mathcal{T}\dentails A \twiddle B\}$. $\mathcal{T}$ is \emph{contextually closed} if it contains every
	conditional that is contextually entailed, i.e. $\mathcal{T}= \{A \twiddle B \mid A,B \subseteq M \tand \mathbb{R}, \mathcal{T}\dentails A \rightarrow B\}$.
\end{definition}

\begin{definition}
	A set $\mathcal{T}$ of conditionals is \emph{rationally complete} with respect to $\mathbb{T},\Delta$ if and only if every conditional that is in the rational closure of
	$\mathbb{T}, \Delta$ rationally follows from $\mathcal{T}$, i.e., $\mathbb{T},\mathcal{T}$ and $\mathbb{T}, \Delta$ define the same entailment relation.
\end{definition}

Is it possible to find a $\mathcal{T}$ such that $\mathcal{T}$ is a compact form of $\Delta$. I.e., it should produce the same ranking on $\mathbb{T}$.

\begin{definition}
	A set $\mathcal{T}$ of conditionals is \emph{contextually complete} with respect to $\mathbb{R}, \Delta$ if and only if every implication that is in the contextual closure of
	$\mathbb{R}, \Delta$ rationally follows from $\mathcal{T}$.
\end{definition}

A set of conditionals that is contextually complete to a ranked context $\mathbb{R}$ and constraint set $\Delta$ provides abstracted perspective, in terms of defeasible
conditionals, on the information contained in the pair $(\mathbb{R}, \Delta )$. It is now the goal, as was done in \Cref{subsubsection:implication-bases}, to find such a set that
is non-redundant, or the \textit{rational basis} of $(\mathbb{R},\Delta)$.

\begin{definition}
	\label{definition:defeasible-closure}

	Let $\mathcal{T}$ be a set of defeasible conditionals over $M$. The operator $X \mapsto \Cn{\twiddle}(X)$ given by $X \cup \{Y \subseteq M \mid X \twiddle Y\}$ is called the \emph{defeasible
	closure} of $X$. It is idempotent, extensive, and non-monotonic.
\end{definition}

\textcolor{red}{
\begin{remark}
	If $\mathcal{T}$ were the defeasible basis we would want $\Cn{\twiddle}(X) = (\underline{X^{\downarrow}})^{\uparrow}$
\end{remark}
}

\subsection{Rational Closure}

\subsection{A Basis for Ranked Contexts}

\subsection{Lexicographic Closure}