\chapter{Non-Monotonic Reasoning}
\label{chapter:non-monotonic-reasoning}

In the previous chapter the matter of consequence was discussed. It is perhaps more appropriate to rephrase this discussion more precisely as being on the matter of classical consequence, for the notion put forward likely does not align with a common understanding of consequence. To make this clearer, consider the following propositions:
\begin{enumerate}
     \item $\texttt{human} \rightarrow \texttt{chronological time}$
     \item $\texttt{billy pilgrim} \rightarrow \texttt{human}$
     \item $\texttt{billy pilgrim} \rightarrow \neg \texttt{chronological time}$
\end{enumerate}
The only models of propositions 1--3 are valuations which map \texttt{billy pilgrim} to false. As such, $\neg \texttt{billy pilgrim}$ is a classical consequence of propositions 1--3: a result of classical logic which is perhaps quite surprising. 

