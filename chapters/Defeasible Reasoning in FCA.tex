\chapter{Defeasible Reasoning in Formal Concept Analysis}
\label{chapter:defeasible-reasoning-in-fca}

The conclusion of \Cref{chapter:defeasible-reasoning-in-fca} marks the completion of the scaffolding of this work. In the
remaining chapters, the more novel research contributes of this work will be discussed. We begin by formalising an approach
to the KLM style of defeasible reasoning within the setting of FCA.

Fortunately (perhaps by design), the context switch is fairly negligible. As we progress, the reasons behind the
selection of the KLM framework as an ideal candidate for defeasible reasoning in FCA should become apparent. Indeed,
understanding the proceeding chapter(s) is benefitted by frequently ``pattern-matching'' between the definitions presented
here, and their counterparts from prior chapters; and so we encourage---and will attempt to facilitate---these frequent backward
references.

\section{Motivation}
\label{section:motivation}

There are many perspectives one may adopt as lense with which to understand the purpose of FCA. Certainly, some of these
perspectives should result in strong arguments against accepting non-monotonicity. Likewise, there are several, distinct,
arguments in favour. One of the most compelling reasons for the adoption of defeasibility in FCA follows from the argument
put forward by Wille \cite{Wille2005}, where the view that concepts are frequently \textit{personal} or \textit{prototypical}
units is discussed. The recognition that defeasible reasoning facilitates a kind of prototypicality is the basis for \Cref{chapter:rational-concepts},
and so we will say no more until then.

\clearpage

\section{Preferential and Ranked Contexts}
\clearpage

\section{Finding order}
\clearpage
This phenomenon was already known to the ancient Greeks, who used the term enthymeme to refer to an argument in which one
or more premises are left implicit. That is the idea that we develop in this section. Also called \textit{expectations}.
\cite{makinson2003bridges}.

\section{Rational Entailment}
\clearpage