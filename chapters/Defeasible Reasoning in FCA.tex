\chapter{Defeasible Reasoning in Formal Concept Analysis}
\label{chapter:defeasible-reasoning-in-fca}

The conclusion of \Cref{part:1} marks the end of the scaffolding component of this work. The remaining chapters are devoted to the novel contributions of this
work: the unification of the ideas presented in \Cref{chapter:defeasible-reasoning,chapter:formal-concept-analysis}. That is, to develop an approach to
defeasible reasoning in FCA in the style of rational consequence.

Fortunately, the context switch is quite natural. As we progress, the reasons behind the selection of the KLM framework as an ideal candidate for defeasible
reasoning in FCA should become apparent. Indeed, understanding the proceeding chapter(s) is benefitted by frequently ``pattern-matching'' between the
definitions presented here, and their counterparts from prior chapters. So we encourage---and will attempt to facilitate---these frequent backwards-facing
references.

This chapter draws on two publications,~\cite{Carr2024} and \cite{Carr2025}.

\section{Motivation}
\label{section:motivation}

The violation of monotonicity in FCA introduces peculiarities which can significantly undermine parts of the theory underlying FCA. For instance, one
perspective on concept lattices is that they describe chains of inheritance from concepts to their subconcepts. This idea becomes difficult to conceptualise
without monotonicity.

Nevertheless, there are certain settings where violation of monotonicity extends the utility of FCA in a reasonable way. The justifications for this are
somewhat similar to those given in \Cref{section:nmr-background}: frequently there are exceptional instances of some class, and it is useful to be able to
sometimes reason in a way which considers these exceptions, and at other times reason prototypically. That is, to reason in spite of the exceptions. There is
an extension of this argument \cite{Belohlavek2022,Wille2005} which makes a stronger case for the prototypical view of concepts; but we will leave this
discussion for the next chapter where it is more relevant.

Consider the context, which will be uses as a running example throughout this chapter.
\begin{figure}[H]
	\hspace{-8.2em}
	\resizebox{1.2\columnwidth}{!}{%

		\begin{cxt}
			\centering
			\cxtName{\textbf{\texttt{lichens}}}
			\atr{\texttt{High-light}}
			\atr{\texttt{low-light}}
			\atr{\texttt{high-temperature}}
			\atr{\texttt{moderate}}
			\atr{\texttt{low-temperature}}
			\atr{\texttt{high-moisture}}
			\atr{\texttt{low-moisture}}
			\atr{\texttt{bark-substrate}}
			\atr{\texttt{rock-substrate}}
			\atr{\texttt{soil-substrate}}
			\atr{\texttt{Plant-debris}}
			\atr{\texttt{moss-substrate}}
			\atr{\texttt{leaf-substrate}}
			\atr{\texttt{constructed-materials-substrate}}
			\atr{\texttt{thallus-crustose}}
			\atr{\texttt{thallus-fruticose}}
			\atr{\texttt{thallus-foliose}}
			\atr{\texttt{thallus-byssoid}}
			\atr{\texttt{orange}}
			\atr{\texttt{yellow}}
			\atr{\texttt{red}}
			\atr{\texttt{pink}}
			\atr{\texttt{beige}}
			\atr{\texttt{brown}}
			\atr{\texttt{gray-white}}
			\atr{\texttt{blue-gray}}
			\atr{\texttt{blue-black}}
			\atr{\texttt{green}}

			\obj{X..X.X..X.....X...X.X.......}{\texttt{Caloplaca-marina}}
			\obj{XX.X.XXXX....X..X.XX........}{\texttt{Xanthoria-parietina}}
			\obj{.XX..X.XX...X..X...X........}{\texttt{Ramalina-celastri}}
			\obj{X..X..X...X....X..XX.......X}{\texttt{Teloschistes-chrysophthalmus}}
			\obj{X.X...X..X...XX......XX.X...}{\texttt{Psora-crenata}}
			\obj{X.XX.X.XX.......X.....X....X}{\texttt{Crocodia-aurata}}
			\obj{............................}{\texttt{*}}
			\obj{X.X.X.X..X...............X.X}{\texttt{Caeruleum-terricola}}
			\obj{XX.X.X.XX.......X....XX..XX.}{\texttt{Sticta-beauvoisii}}
			\obj{X.XX..X.X.......X...X..X....}{\texttt{Lasallia-rubiginosa}}
			\obj{X...X.X.XX......X......X.X..}{\texttt{Leptogium-menziesii}}
			\obj{X..X..X.XX.X....X......X.X..}{\texttt{Scytinium-gelatinosum}}
			\obj{XX..X.X.XX.X..X.........X...}{\texttt{Lepraria-borealis}}
			\obj{XX..XX.XXXXX..X.........XX..}{\texttt{Lepraria-caesiella}}
			\obj{XX.X.X.XX.....X....X........}{\texttt{Chrysothrix-candelaris}}
			\obj{X...XX....X......X......X...}{\texttt{Roccellinastrum-lagarostrobi}}
			\obj{X.X..X....X......X......X...}{\texttt{Roccellinastrum-spongoideum}}
			\obj{X.X..X.X...XX.X.......X.X...}{\texttt{Thelenella-indica}}
		\end{cxt}}
	\caption{A formal context of lichens}
	\label{fig:lichens}
\end{figure}

The context contains several different species of lichen, described by certain morphological and ecological properties. Of course, the context is by no means
exhaustive in either dimension. If the task is to discover patterns that exist between properties---represented by attribute implications---the brittleness of
these implications is a significantly limiting factor on which patterns may be learned. The existence of a single counter-example to an otherwise acceptable
correspondence between attributes is fatal. Compounding on this problem, there may be errors in the context which present false counter-examples.

\subsection{Association Rules}
\label{subsection:association-rules}

Concerns over brittleness in the semantics of (classical) attribute implications are well known in the FCA literature, and in data mining more generally
\cite{Lakhal2005}. \textit{Association rules} are able express implication-like statements despite the existence of counterexamples in the data, or in our case
a context. These are statements like \say{$66\%$ of the members of congress who were \textit{Democrats}, which makes up $50\%$ of congress, voted on the
	\textit{MX-Missile} bill}.

The \emph{support} of a set of attributes $X \subseteq M$ in a context $\GMI$ is the fraction of objects which have all attributes $X$ over the total number of
objects; and so $\mathrm{supp}( X) = \nicefrac{|X^{\downarrow}|}{|G|}$, the (relative) support of a rule $A \Rightarrow B$ is simply the support of $A$. The
\textit{confidence} of a rule is a measure of how frequently the premise and conclusion occur, relative to how frequently the premise occurs, formally
$\mathrm{conf}(A \Rightarrow B) = \nicefrac{\mathrm{supp}(X \cup Y)}{\mathrm{supp}(X)}$. In the example above, the support of the rule is $50\%$ and its
confidence is $66\%$. Association rules which meet specified lower-thresholds for support and confidence are then accepted.

While association rules present a reasonable solution to the problem of discovering and representing implications with exceptions, there are some undesirable
qualities. Settling on the threshold vallues for support and confidence is usually dependent on human input, and there is no obvious intuition for how one
might arrive at these. Frequently, it is a requirement to know the data well, which creates a kind of circularity as the purpose of this work is to extend a
tool which helps one understand their data.

The situation gets worse when we try to describe the properties of reasoning with association rules. Consider the two rules below, which should be accepted
with $\mathrm{minconf}= 0.6$ and $\mathrm{minsup}= 0.5$.

\begin{align*}
	\texttt{dem}\Rightarrow \texttt{mxm} \\
	% ; \text{with}\; \mathrm{Sup}= 0.5 \; \text{and}\; \mathrm{Conf}= 0.66 \\
	\texttt{dem}\Rightarrow \texttt{dfe} % \; \text{with}\; \mathrm{Sup}= 0.5 \; \text{and}\; \mathrm{Conf}= 0.66
\end{align*}
In natural language, we might summarise this scenario as saying \say{Democrats voted in favour of the \textit{MX-Missile} and \textit{Duty-Free
		Exports}}. To make future analysis simpler, we could suggest that whenever we accept association rules of the form
\begin{align*}
	\phi \Rightarrow \psi \\
	\phi \Rightarrow \gamma,
\end{align*}
we may compress the information as the simpler representation:
\begin{align*}
	\phi \Rightarrow \psi \land \gamma
\end{align*}

Unfortunately, this is not always the case. In the above scenario, the rule $\texttt{dem}\Rightarrow \{\texttt{mxm,dfe}\}$ has a confidence of $0.47$, which is
below the $\mathrm{minconf}$ threshold.

That the pattern of reasoning enforced by association rules does not satisfy the above characterisation---which is the same property as was discussed at
\Cref{postulate:and}---may be overlooked in isolation. The more general problem is that association rules do not admit description via \lucas{the kind of
	axiomisation we want [source needed]}.

\section{Preferential Contexts}
\label{section:preferential-contexts}

We begin the introduction of KLM style defeasible reasoning to FCA by defining a structure that is in almost perfect analogy to preferential interpretations.
The analogy is not perfect as we initially behave quite conservatively and attempt to make as few changes as possible to the default setting of FCA. As we get
further along in the discussion, more substantial amendments to the default setting are suggested. These amendments serve to make the faithful the analogy. The
idea behind the order of this exposition is that we view this work largely as extending FCA with ideas borrowed from defeasible reasoning. Then, there may be
those who find some of the essential parts of defeasible reasoning quite useful in FCA, but wish to remain in the default setting.

With that said, we define the first extension of a \textit{preferential context}.
\begin{definition}
	\label{definition:preferential-context}
	\index{formal context! preferential context}

	A \emph{preferential context} $\pcontext = \PGMI$ is a formal context $\GMI$ where $\prec$ denotes a strict partial order, or \emph{preference order}, on the
	set of objects.
\end{definition}

Preferential contexts, albeit with a different motivation, were first introduced by Obiedkov \shortcite{Obiedkov2012}, and later on with our purposes in mind
in \cite{Carr2024}. They are nothing more than a formal context with a preference order on the set of objects. We provide an example of what such a preference
ordering might look like below (in the interests of brevity, we consider a subcontext of \Cref{fig:lichens} constructed by taking only the first six objects).

\begin{figure}[H]
	\centering
	% \begin{tikzpicture}[every node/.style={rectangle, rounded corners=2pt, draw, text=black, inner sep=2pt, font=\tiny}]
	\begin{tikzpicture}[every node/.style={text=black, inner sep=2pt, font=\small}]
		\node (root) at (0,0) {\shortstack{\texttt{Caloplaca-}\\\texttt{marina}}};
		\node (1) at (-2,1.5) {\shortstack{\texttt{Xanthoria-}\\\texttt{parietina}}};
		\draw (root) -- (1);
		\node (2) at (0,1.5) {\shortstack{\texttt{Ramalina-}\\\texttt{celastri}}};
		\draw (root) -- (2);
		\node (3) at (2,1.5) {\shortstack{\texttt{Teloschistes-}\\\texttt{chrysophthalmus}}};
		\draw (root) -- (3);
		\node (12) at (-4,3) {\shortstack{\texttt{Psora-}\\\texttt{crenata}}};
		\draw (1) -- (12);
		\node (13) at (-1,3) {\shortstack{\texttt{Crocodia-}\\\texttt{aurata}}};
		\draw (1) -- (13);
	\end{tikzpicture}
	\caption{A preference relation on a subcontext of \Cref{fig:lichens}}
	\label{figure:preference-relation-lichen}
\end{figure}

Recall that preferential interpretations were defined by a relation on states which were then mapped to valuations, with no requirement that the mapping be
injective. It may then be surprising that states seem to have been entirely omitted from our definition. The reason being that formal contexts have no
restriction on the inclusion of two objects with an equivalent intension, meaning that states seem to be implicitly constructed in contexts.

\begin{lemma}
	\label{lemma:objects-preferential}

	There exists a preferential context $\PGMI$ where there exist two objects, $g,h \in G$, with $g^{\uparrow}= h^{\uparrow}$ that defines the consequence relation
	$\twiddle_{\pin}$ such that no preferential context $\pcontext'$ without `duplicate' objects defines the same relation.
\end{lemma}

Under the preference relation, we describe the $\prec$-minimal states as one might expect, through the process of \textit{minimisation}.
\begin{definition}
	\label{definition:minimisation}

	In a preferential context $\pcontext = \PGMI$, the \emph{minimisation} of a set $A \subseteq G$ is defined by
	\[
		\underline{A}\coloneqq \{\, g \in A \mid \nexists h \in A : (h \prec g) \,\}.
	\]
\end{definition}

Frequently, the minimisation operator is applied to a set of objects that is the result of previous derivation from a set of attributes. That is, we might
begin with the attribute set $\{\, \texttt {rock-substrate}\,\}$ and wish to find the most preferred objects that have this attribute. We call this the
\textit{minimised-derivation}, and write $\Mind{\{\,\texttt{rock-substrate}\,\}}$. If one were to apply a second derivation to the set resulting from a
minimised-derivation, the result represents all those attributes shared by the preferred objects satisfying the initial set. In turn, we refer to this as
\textit{minimised-return} and write $\Minr{\{ \, \texttt{rock-substrate} \, \}}$.

The minimised-derivation of $\{\, \texttt{rock-substrate}\, \}$ yields \{\texttt{Caloplaca-Marina}\}. One interpretation of this is that
\texttt{Caloplaca-Marina} has some rock-substrate related property that sets it apart from all other species of lichen that are also have a rock-substrate.
There is some subtlety here: the status of \texttt{Caloplaca-Marina} as the uniquely preferred species of lichen with a rock-substrate may well be the
byproduct of some unrelated preference. For example, the preference order might be that red lichens are preferred to yellow ones, who are in turn preferred to
any other colours. The point we are making has little material impact, but is rather important for ones' intuition of what preference relations represent, or
how they may be constructed. For the time being, we remain agnostic to this latter question---how they might be constructed. This will be the subject of
\Cref{subsubsection:finding-order}.

Preferential contexts allow us to speak about \textit{defeasible conditionals} in the setting of FCA. Much like classical implications in FCA, defeasible
conditionals are defined over an attribute set $M$, and so the expression $A \twiddle B$ should be understood to be saying that the ``typical'' objects which
have all attributes from $A$ have all attributes from $B$. Defeasible conditionals are satisfied with respect to a preferential context, admitting the
following definition.

\begin{definition}
	\label{definition:satisfaction-preferential-context}

	Let $\pcontext = \PGMI$ be a preferential context, $A,B \subseteq M$ subsets of attributes, and $A \twiddle B$ a \emph{defeasible conditional}. We say that $A
		\twiddle B$ is \emph{satisfied} in $\pcontext$ if and only if $\Mind{A}\subseteq B^{\downarrow}$ and write $\pcontext \vDash_{\pcontext}A \twiddle B$.
\end{definition}

The condition for satisfaction of $A \twiddle B$ is that all the objects in the minimised-derivation of $A$ are members of the derivation of $B$. For those
with a background in FCA, satisfaction may be expressed more familiarly as $B \subseteq \Minr{A}$.

\begin{example}
	The following defeasible conditionals are all satisfied by the preferential context composed from \Cref{fig:lichens} and \Cref{figure:preference-relation-lichen}
	\begin{enumerate}
		\item $\{\texttt{high-light}\} \twiddle \{\texttt{red}\}$
		\item $\{\texttt{high-light,low-moisture}\} \ntwiddle \{\texttt{red}\}$
		\item $\{\texttt{high-light, yellow}\} \twiddle \{\texttt{orange}\}$
		\item $\{\texttt{orange}\} \twiddle \{\texttt{red}\}$
		\item $\{\texttt{high-light,yellow}\} \ntwiddle \{\texttt{red}\}$
	\end{enumerate}

	These conditionals tell us that usually species of lichen that live in areas with a signficant amount of light are red, but those which additionally live in
	areas with low-moisture usually are not red. While it is typical for yellow species of lichen that live in well-lit areas to also be orange, and it is also
	typical for orange species of lichen to have red colouring too, it does not follow that yellow species of lichen living in well-lit areas have red. In other
	words, the logic is non-monotonic and intransitive.
\end{example}

It might be suspected that the kind of reasoning described by preferential contexts corresponds to the preferential consequence relations discussed earlier. We
show that this is not quite true.

\begin{theorem}
	\label{theorem:preferential-context-partial-soundness}

	The consequence relation $\twiddle_{\pcontext}$ induced by a preferential context $\PGMI$ satisfies \textit{Reflexivity, Left-logical equivalence, Right
		weakening, And,} and \textit{Cautious Monotony}.
\end{theorem}

\begin{proof}
	\label{proof:preferential-context-partial-soundness}

	Let $\pcontext = \PGMI$ be a preferential context, and $A,B,C,D \subseteq M$ sets of attributes.

	For \textit{Reflexivity}, it needs to be shown that $\pcontext \vDash A \twiddle A$, equivalently that $\Mind{A}\subseteq A^{\downarrow}$. This holds by
	definition of the minimal-derivation operator. \lucas{Make sure this holds if $A^{\downarrow}= \emptyset$}

	For \textit{Left-logical equivalence}, if $\pcontext \vDash_{\pcontext}A \twiddle C$ and $A = B$, then $\pcontext \vDash_{\pcontext}B \twiddle C$. Assume that
	$\pcontext \vDash A \twiddle C$ and $A = B$ (by usual set equality). By this assumption, (i) $\Mind{A}\subseteq C^{\downarrow}$ and (ii) $\Mind{A}= \Mind{B}$
	both hold. It is then clear that we can make the substitution that results in $\Mind{B}\subseteq C^{\downarrow}$, which is equivalent to $\pcontext \vDash B
		\twiddle C$.

	For \textit{Right weakening} we show that if $\pcontext \vDash A \rightarrow B, B \twiddle C$ then $\pcontext \vDash A \twiddle C$.
\end{proof}

The failure to satisfy the \textit{Or} postulate is due more to a technical point than a counter example. The assumed setting, of FCA's attribute logic, is
restricted to definite Horn clauses, and therefore not sufficiently expressive for either negation or disjunction. We may address this issue by consdering a
more expressive variant of attribute logic: the compound attributes discussed in \Cref{subsection:compound-attributes}. A minimal change is to simply extend
the language of defeasible conditionals to compound attributes; since these more expressive formulae are constructed from a (normal) set of attributes, the
definition of a preferential context remains the same.

\begin{definition}
	\label{definition:defeasible-conditionals-compound}

	Let $\pcontext = \PGMI$ be a preferential context. Then $\phi \twiddle \psi$ where $\phi,\psi \in M^{+}$ is a \emph{defeasible conditional} over the compound
	attributes of $M$.
\end{definition}

We re-emphasize the point, which was originally made in \Cref{section:klm-framework}, that although the formulae that make up the antecedent and consequent of
a defeasible conditional $\phi \twiddle \psi$ are in the language $M^{+}$, they are not themselves allowed to be defeasible conditionals. Thus, nesting of
defeasible conditionals—expressions such as $\phi \twiddle \gamma \twiddle \psi$—is not permitted. Fortunately, with the relatively minor extension of compound
attributes, the logic is sufficiently expressive for disjunction, and so we construct the following representation result.

\begin{theorem}
	\label{theorem:preferential-context-soundness}

	The consequence relation $\twiddle_{\pcontext}$ induced by a preferential context $\pcontext = \PGMI$ with defeasible conditionals over the compound attributes
	of $M$ is a preferential consequence relation.
\end{theorem}

\begin{proof}
	\label{proof:preferential-context-soundness}
\end{proof}

\begin{theorem}
	\label{theorem:preferential-context-completeness}
	Let $\twiddle_{P}$ be an arbitrary preferential consequence relation over compound attributes $M^{+}$. Then there exists a preferential context
	$\pcontext = \PGMI$ inducing the consequence relation $\twiddle_{\pcontext}$ such that $\twiddle_{P}$ is precisely $\twiddle_{\pcontext}$.
\end{theorem}

\begin{proof}

	\label{proof:preferential-context-completeness}
\end{proof}

\begin{proposition}
	\label{proposition:preferential-context-irrational}

	Preferential contexts do not satisfy rational monotony.
\end{proposition}

% \begin{example} \label{example:counter-example-rationality} \end{example}

\begin{definition}
	\label{definition:ranked-context}
	A \emph{ranked context} $\rcontext$ is a preferential context $\RGMI$ where the preference relation $\mathsf{R}$ on objects is a \textit{strict weak order}.
\end{definition}

\subsubsection{Finding Order}
\label{subsubsection:finding-order}

Thus far, the discussion on finding a suitable preference relation has been omitted. It is a non-trivial task to compare objects in a context.

\begin{algo}{\textsc{ObjectRank}}
	\label{algorithm:ObjectRank}
	\Require A set $\Delta$ of defeasible conditionals over an attribute set $M$ \Require A formal context $\GMI$
	\Ensure A ranking $\mathsf{R}\colon G \to \mathbb{N}$ such that $\Delta$ holds in $\RGMI$ if $\GMI$ is $\Delta$-compatible; $\bot$, otherwise.
	\State $i \coloneq 0$ \State \textit{initialise} $\mathsf{R}(g) \coloneq \vert G \vert$ for all $g \in G$ \State $\Gamma \coloneq \Delta$
	\While{$\exists \, g \in G \colon \mathsf{R}(g) = \vert G \vert$}
	\ForAll{$g \in G$ with $\mathsf{R}(g) = \vert G \vert$}
	\If{$g \vDash \Gamma$} \textit{update} $\mathsf{R}(g) = i$ \EndIf
	\If{$\forall \, g \in G \colon \mathsf{R}(g) \neq i$} \State \Return $\bot$ \EndIf \State $\Gamma \coloneq \{(\phi \twiddle \psi) \in \Gamma \mid g \nvDash \phi \; \text{for all}\; g \in G
		\; \text{with}\; \mathsf{R}(g) = i\}$ \State $i \coloneq i + 1$ \EndFor \EndWhile
	\State \Return $\mathsf{R}$
\end{algo}

\section{Entailment}
\subsection{Contextual Rational Closure}

\begin{definition}
	A conditional $A \twiddle B$ \emph{rationally follows} from a set $\mathcal{T}$ of conditionals if and only if $\mathbb{T}, \mathcal{T}\dentails A \twiddle B$, where $\mathbb{T}$ is the \emph{test
		context}. It \emph{contextually follows} when $\mathbb{R}, \mathcal{T}\dentails A \rightarrow B$.
\end{definition}

\begin{definition}
	A set $\mathcal{T}$ is \emph{closed} with respect to a pair if and only if it contains every conditional that is rationally entailed. That is, if
	$\mathcal{T}= \{A \twiddle B \mid A,B \subseteq M \tand \mathbb{T},\mathcal{T}\dentails A \twiddle B\}$. $\mathcal{T}$ is \emph{contextually closed} if it contains every conditional that
	is contextually entailed, i.e. $\mathcal{T}= \{A \twiddle B \mid A,B \subseteq M \tand \mathbb{R}, \mathcal{T}\dentails A \rightarrow B\}$.
\end{definition}

\begin{definition}
	A set $\mathcal{T}$ of conditionals is \emph{rationally complete} with respect to $\mathbb{T},\Delta$ if and only if every conditional that is in the rational closure of
	$\mathbb{T}, \Delta$ rationally follows from $\mathcal{T}$, i.e., $\mathbb{T},\mathcal{T}$ and $\mathbb{T}, \Delta$ define the same entailment relation.
\end{definition}
Is it possible to find a$\mathcal{T}$such that$\mathcal{T}$is a compact form of$\Delta$. I.e., it should produce the same ranking on$\mathbb{T}$.

\begin{definition}
	A set $\mathcal{T}$ of conditionals is \emph{contextually complete} with respect to $\mathbb{R}, \Delta$ if and only if every implication that is in the contextual closure of
	$\mathbb{R}, \Delta$ rationally follows from $\mathcal{T}$.
\end{definition}

A set of conditionals that is contextually complete to a ranked context$\mathbb{R}$and constraint set$\Delta$provides abstracted perspective, in terms of
defeasible conditionals, on the information contained in the pair$(\mathbb{R}, \Delta )$. It is now the goal, as was done in
\Cref{subsubsection:implication-bases}, to find such a set that is non-redundant, or the \textit{rational basis}of$(\mathbb{R},\Delta)$.

\begin{definition}
	\label{definition:defeasible-closure}

	Let $\mathcal{T}$ be a set of defeasible conditionals over $M$. The operator $X \mapsto \Cn{\twiddle}(X)$ given by $X \cup \{Y \subseteq M \mid X \twiddle Y\}$
	is called the \emph{defeasible closure} of $X$. It is idempotent, extensive, and non-monotonic.
\end{definition}

\textcolor{red}{
	\begin{remark}
		If $\mathcal{T}$ were the defeasible basis we would want $\Cn{\twiddle}(X) = (\underline{X^{\downarrow}})^{\uparrow}$
	\end{remark}
}

\subsection{Rational Closure}

\subsection{A Basis for Ranked Contexts}

\subsection{Lexicographic Closure}
