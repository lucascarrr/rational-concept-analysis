\chapter{Rational Concepts}
\label{chapter:rational-concepts}

\section{Motivation}
\begin{figure}[H]
	\centering
	\unitlength 1.75mm
	\begin{diagram}{50}{100}

		% Row 1
		\Node{1}{25}{95}

		\Node{2}{32.5}{84} \Edge{1}{2} \rightAttbox{2}{0}{1}{wpcs}

		\Node{3}{10}{75} \Edge{1}{3} \leftAttbox{3}{0}{1}{crm,esa,pff}

		\Node{4}{32.5}{62} \Edge{2}{4} \rightAttbox{4}{0}{1}{easa}

		\Node{5}{45}{60} \Edge{2}{5} \rightAttbox{5}{0}{1}{dem}

		\Node{6}{4}{58} \Edge{3}{6} \leftAttbox{6}{0}{1}{rep,edu,srs}

		\Node{7}{17}{58} \Edge{3}{7} \Edge{2}{7} \leftAttbox{7}{1}{1}{rgs}

		\Node{8}{0}{47} \Edge{6}{8} \leftAttbox{8}{1}{1}{imm}
		\leftObjbox{8}{1}{1}{moc-79}

		\Node{9}{40}{46} \Edge{4}{9} \Edge{5}{9} \leftAttbox{9}{1}{1}{dfe}

		\Node{10}{50}{46} \Edge{5}{10} \rightObjbox{10}{1}{1}{moc-62}

		\Node{11}{10}{40} \Edge{6}{11} \Edge{7}{11}
		\rightObjbox{11}{1}{1}{moc-1}

		\Node{12}{22.5}{43} \Edge{7}{12} \Edge{4}{12}

		\Node{13}{45}{33} \Edge{9}{13} \Edge{10}{13}
		\rightObjbox{13}{1}{1}{moc-34}

		\Node{14}{30}{28} \Edge{9}{14} \Edge{12}{14}
		\leftObjbox{14}{1}{1}{moc-6}

		\Node{15}{12.5}{23} \Edge{8}{15} \Edge{11}{15} \Edge{12}{15}
		\leftObjbox{15}{1}{1}{moc-0}

		\Node{16}{25}{10} \Edge{13}{16} \Edge{14}{16} \Edge{15}{16}
		\leftAttbox{16}{2}{0}{scc}
	\end{diagram}
	% \vspace{-6em}
	\caption{The concept lattice corresponding to \Cref{cxt:grouplikes}}
	\label{figure:concept-lattice-congress}
\end{figure}
\section{Background}

\section{Preconcepts}
FCA Theory.

\section{Naive Rational Concepts}

I think here it would be a good place to discuss the typical concepts
a la \cite{Carr2024}. Can go into detail, but also then discuss why
this doesn't work. It relates to the fact that transitivity is not a
property of this logic, but lattices rely on transitivity.

Could include discussion on defeasible inheritance networks; but this
is not really relevant because I couldn't get it to work. Maybe worth
mentioning in a closing paragraph; i.e., \say{This is what was tried,
	it didn't work because of X. Perhaps it is worth exploring some
	non-monotonic lattice theory, but this was beyond the scope of this
	thesis.}

\section{Version II}

\section{Relation to Contextual Rational Closure}
Here we want to show how the information you get from these typical
concept lattices corresponds to the information you get from rational
closure. It's a bit obvious, but pointing it out is maybe the purpose
of this work, since its all a bit obvious.
