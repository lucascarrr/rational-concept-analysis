\chapter{Defeasible Reasoning}
\label{chapter:defeasible-reasoning}

Non-monotonicity in logical systems has been the focus of study for decades, and several distinct formalisms have been developed. The motivation is to expand the inference power of logical systems, beyond the classical, to a more credulous one. This work is interested primarily in non-monotonic reasoning in the style of preferential semantics due to Shoham \cite{shohamSemanticApproach}, which benefits from a well developed model theory for which there is an intuitive analogue to notinos central to FCA. In particular, we are interested in the framework put forward by Kraus, Lehmann, and Magidor \cite{kraus1990nonmonotonic,lehmann1992what}, commonly initalised as the `KLM-framework'.

In the proceeding, we provide a background on non-monotonic reasoning in general, before introducing the KLM framework. We assume all the same conventions introduced in \Cref{section:propositional-logic} for the technical discussions in this chapter.
\section{Background on Non-monotonic Reasoning}
\label{section:nmr-background}
Near the end of the previous chapter, the matter of consequence was discussed in a very formal sense. It is perhaps helpful to distinguish this subject---classical consequence---from the common concept, as the former yields some surprising results which do not appear at all congruent with how a person, or otherwise intelligent agent should reason \cite{tarski1936consequence, kraus1990nonmonotonic}. For a demonstration of a result which may be \textit{surprising}, consider the following propositions:

\begin{enumerate}
     \item $\texttt{human} \rightarrow \texttt{chronological time}$
     \item $\texttt{soldier} \rightarrow \texttt{human}$
     \item $\texttt{billy pilgrim} \rightarrow \neg \texttt{chronological time}$
\end{enumerate}

Knowing these propositions, if we were to encounter an individual in fatigues we might find it sensible---by propositions 2 and 1---to infer that the individual experienced time chronologically. If we were to later learn that the individual was, in fact, Billy Pilgrim, given proposition 3, we should like to retract our prior inference, replacing it with the knowledge that the individual does not experience time chronologically.

Such recourse is not, however, possible: When we see the that the individual is a soldier, the possible worlds satisfying our knowledge are reduced to the single model: $\{\overline{b},c,h,s\}$ (where $b$: \texttt{billy pilgrim}, $c$: \texttt{chronological time}. $h$: \texttt{human}, and $s$: \texttt{soldier}). Should we later learn that the individual is indeed Billy Pilgrim, the theory becomes inconsistent.
%TODO: Add section in preliminaries about principle of explosion
And, by the principle of explosion, discussed in \Cref{subsection:logical-consequence}, our theory now entails that the individual experiences time both chronologically and non-chronologically.

This property of classical logic---that adding new information never results in retraction of pre-existing knowledge---is called monotonicity. Monotonicity requires that when we make a claim like \say{humans experience time chronologically}, we must be absolutely sure of ourselves, so as to never worry about needing to retract an inference. Of course, this is too strict a requirement as we cannot determine for all future, present, and past humans if this were the case. If we remain in the classical realm, it seems our only options are to abandon our original claim or risk explosion.

At this point it is a good idea to provide some clarification on how we might formally approach this issue. Continuing with the same example---and continuing to allow ourselves to entertain the idea of experiencing non-chronological time---it would certainly be agreed that typically soldiers are human, and also that typically humans experience time chronologically. To resolve that Billy Pilgrim is a soldier, and therefore a human, who does not experience chronological time, we need only to point out that he is an atypical human.

To make the previous paragraph more formal, we remind the reader of the discussion held around \Cref{definition:logical-consequence}: A formula $\phi$ is a logical consequence of a set $\Gamma$ thereof if every model of $\Gamma$ is also a model of $\phi$. Put differently, there is no valuation (or, possible world) where $\Gamma$ is true and $\phi$ is false. It follows directly that $\phi$ remains a logical consequence of $\Gamma \cup \{\psi\}$, since $\Gamma$ is true in any world where $\Gamma \cup \{\psi\}$ is true.

It was pointed out by Shoham \cite{shohamSemanticApproach} that we can restrict semantic consideration to a priveledged subset of models which are deemed ``preferable''; we call these selected models the ``minimal'' models (a choice which will make more sense as this chapter progresses).

\section{Preferential Reasoning}
\label{section:preferential-reasoning}
The KLM framework for non-monotonic reasoning was initially described by a collection of consequence relations satisfying certain properties---frequently called the \textit{rationality postulates}---with each system in the collection being progressively stronger than its predecessor. A consequence relation in the KLM framework is denoted with a `$\twiddle$' rather than the usual `$\vdash$', which we call `twiddle'. Then, $\phi \twiddle \psi$ should be understood to say \say{if $\phi$, then typically also $\psi$}. Such a pair is called a \textit{conditional assertion}; as we may expect $\phi \twiddle \psi$ means $(\phi,\psi) \in \; \twiddle$ and $\phi \ntwiddle \psi$ means $(\phi, \psi) \not \in \; \twiddle$.

We may, at times of potential confusion, use a subscript to disambiguate which consequence relation is being referred to, and so $\twiddle_C$ would refer to a cumulative relation, as defined below.

\subsection{System C}
\label{subsection:system-c}
\textit{Cumulative} consequence relations (shortened to System C) represent the weakest of the systems in the KLM framework, and accordingly present a natural starting point \cite{kraus1990nonmonotonic}.

\begin{definition}
\label{definition:cumulative-consequence-relation}
The consequence relation $\twiddle$ is a \emph{cumulative consequence relation} if and only if it satisfies the properties of \emph{Reflexivity, Left Logical Equivalence, Right Weakening, Cut} and \emph{Cautious Monotony}.
\end{definition}

\begin{align}
\text{(Reflexivity)} & \quad \alpha \twiddle \alpha
\end{align}


\begin{align}
\text{(Left Logical Equivalence)} & \quad \alpha \leftrightarrow \beta \twiddle \alpha
\end{align}


\subsection{System P}
\label{subsection:system-P}
